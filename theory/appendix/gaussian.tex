\section{Gaussian potential}

\noindent
Gaussian distribution:

\begin{equation}\label{Gaussian-distribution}
    G(\boldsymbol{r}) = \left( \frac{\alpha}{\sqrt{\pi}} \right)^{3} e^{-\alpha^2|\boldsymbol{r}|^2}
\end{equation}

\noindent
Coulomb potential generated by a charge of Gaussian distribution(omit $\frac{e^2}{4\pi\epsilon_0}$):

\begin{equation}\label{Gaussian-potential}
    \begin{gathered}
        \nabla^{2} \phi(\boldsymbol{r}) = - 4\pi G(\boldsymbol{r}) 
        \Rightarrow \phi(\boldsymbol{r}) = \frac{erf(\alpha |\boldsymbol{r}|)}{|\boldsymbol{r}|} \\
        \phi(0) = \frac{2\alpha}{\sqrt{\pi}} ,\,
        erf(x) = \frac{2}{\sqrt{\pi}} \int_{0}^{x} e^{-t^2} \mathrm{d}t 
    \end{gathered}
\end{equation}

\noindent
Features: long-range bahavior is is proportional to $\frac{1}{r}$.

Figure


\begin{equation}
    \lim_{|\boldsymbol{r}| \rightarrow \infty} \phi(\boldsymbol{r}) = \frac{1}{|\boldsymbol{r}|}
\end{equation}

\subsection{Fourier transformation}

\noindent
3D case:

\begin{equation}\label{Gaussian-potential-Fourier-3D}
    \phi(\boldsymbol{k}) = \int \mathrm{d}\boldsymbol{r} \phi(\boldsymbol{r}) e^{-ikr}
    = \frac{4\pi}{\boldsymbol{k}^2} e^{-\frac{\boldsymbol{k}^2}{4\alpha^2}}
\end{equation}

\noindent
2D case:

\begin{equation}\label{Gaussian-potential-Fourier-2D}
    \phi(k_x,k_y,z) = \int \mathrm{d}x \mathrm{d}y \phi(\boldsymbol{r}) e^{-i(k_xx+k_yy)}
    = \frac{\pi}{|\boldsymbol{k}|} \left[ 
        e^{|\boldsymbol{k}|z} erfc(\frac{|\boldsymbol{k}|}{2\alpha} 
        + \alpha z) + e^{-|\boldsymbol{k}|z} erfc(\frac{|\boldsymbol{k}|}{2\alpha} - \alpha z) 
        \right] 
\end{equation}

\noindent
1D case: To be continued.

