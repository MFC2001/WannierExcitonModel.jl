\documentclass{report}

\usepackage[top=2.5cm,bottom=2.5cm,left=2cm,right=2.5cm,headheight=1cm,footskip=1cm,heightrounded]{geometry}


%---------- 编码与字体 ----------
\usepackage[utf8]{inputenc}   % pdfLaTeX 用,若 Xe/Lua 可删
\usepackage[T1]{fontenc}
\usepackage{lmodern}          % 高质量拉丁字体

%---------- 英文微排版 ----------
\usepackage[english]{babel}
\usepackage{microtype}        % 字距微调,行末更好对齐

%---------- 数学与代码 ----------
\usepackage{amsmath,amsfonts,amssymb}
\usepackage{listings}         % 代码高亮
\usepackage{xcolor}
\lstset{
  basicstyle=\ttfamily\small,
  keywordstyle=\color{blue!70!black},
  stringstyle=\color{red!70!black},
  commentstyle=\color{green!40!black},
  breaklines=true,
  columns=flexible,
  numbers=left,
  numberstyle=\tiny\color{gray}
}

%---------- 图表与浮动 ----------
\usepackage{graphicx}
\usepackage{caption,subcaption}
\usepackage{float}            % 提供 [H] 强制位置

%---------- 交叉引用与超链接 ----------
\usepackage{hyperref}
\hypersetup{
  colorlinks=true,
  linkcolor=blue,
  filecolor=magenta,
  urlcolor=cyan,
  pdftitle={MyPackage Documentation},
  pdfauthor={Your Name},
  pdfsubject={LaTeX package user manual},
  pdfkeywords={latex package documentation}
}

%---------- 页眉页脚 ----------
\usepackage{fancyhdr}
\pagestyle{fancy}
\fancyhf{}
\fancyhead[L]{\nouppercase\leftmark} % 章标题
\fancyfoot[C]{\thepage}
\renewcommand{\headrulewidth}{0.4pt}

%---------- 索引(可选) ----------
\usepackage{makeidx}
\makeindex

%公式中输出正体;
\newcommand{\eqrm}[1]{
	{\rm #1 }
}

%opening
\title{WannierExcitonModel.jl - Theory}
\author{FanchenMeng mengfc011220@gmail.com}



\begin{document}

\maketitle

\tableofcontents

\newpage

\chapter{Introduction}

This is the Physical background for \href{https://github.com/MFC2001/WannierExcitonModel.jl}{WannierExcitonModel.jl}.

\chapter{Tight Binding Model}

\section{Poison summation}

\noindent
Finite case:

\begin{equation}
  \begin{aligned}
    \sum_{\boldsymbol{R}} e^{i(\boldsymbol{k}-\boldsymbol{k}') \cdot \boldsymbol{R}} &= N \delta_{\boldsymbol{k}\boldsymbol{k}'} \\
    \sum_{\boldsymbol{k}} e^{i\boldsymbol{k} \cdot (\boldsymbol{R}-\boldsymbol{R}')} &= N \delta_{\boldsymbol{R}\boldsymbol{R}'} 
  \end{aligned}
\end{equation}

\noindent
Infinite case:

\begin{equation}
  \begin{aligned}
    \sum_{\boldsymbol{R}} e^{i(\boldsymbol{k}-\boldsymbol{k}') \cdot \boldsymbol{R}} = \frac{(2\pi)^3}{V} \delta(\boldsymbol{k}-\boldsymbol{k}') \\
    \frac{V}{(2\pi)^3} \int \mathrm{d}\boldsymbol{k} e^{i\boldsymbol{k} \cdot (\boldsymbol{R}-\boldsymbol{R}')} = \delta_{\boldsymbol{R}\boldsymbol{R}'} 
  \end{aligned}
\end{equation}


\section{Band state}

For the case where the system is infinite, we define

\begin{equation}
  \begin{aligned}
    \text{Orthonormality:} \;&
    \left\langle m\boldsymbol{k} \middle| n\boldsymbol{k}' \right\rangle 
      = \frac{(2\pi)^3}{V} \delta_{mn} \delta(\boldsymbol{k}-\boldsymbol{k}') \\
    \text{Completeness:} \;&
    \hat{\mathcal{I}} = \frac{V}{(2\pi)^3} \int \mathrm{d}\boldsymbol{k} 
      \sum_{n} \left| n\boldsymbol{k} \right\rangle \left\langle n\boldsymbol{k} \right| 
  \end{aligned}
\end{equation}

\noindent
where $\left| n\boldsymbol{k} \right\rangle$ is the single partical Bloch state.
Under unitary transformation $U_{in}^{\boldsymbol{k}}$, these two equations maintain the formation:

\begin{equation}
  \begin{gathered}
    \left| n \boldsymbol{k} \right\rangle 
      = \sum_{i} U_{in}^{\boldsymbol{k}} \left| i\boldsymbol{k} \right\rangle ,\;
    \left| i\boldsymbol{k} \right\rangle = \sum_{n} U_{in}^{\boldsymbol{k}*} \left| n \boldsymbol{k} \right\rangle \\
    \begin{aligned}
      \text{Orthonormality:} \;&
        \left\langle i\boldsymbol{k} \middle| j\boldsymbol{k}' \right\rangle = \frac{(2\pi)^3}{V} \delta_{ij} \delta(\boldsymbol{k}-\boldsymbol{k}') \\
      \text{Completeness:} \;&
        \hat{\mathcal{I}} = \frac{V}{(2\pi)^3} \int \mathrm{d}\boldsymbol{k} \sum_{i} \left| i\boldsymbol{k} \right\rangle \left\langle i\boldsymbol{k} \right| 
    \end{aligned}
  \end{gathered}
\end{equation}

\noindent
We can define Wannier states from Bloch states, and deduce its orthonormality and completeness:

\begin{equation}
  \begin{gathered}
    \left| i\boldsymbol{k} \right\rangle 
      = \sum_{\boldsymbol{R}} e^{i\boldsymbol{k} \cdot \boldsymbol{R}} \left| i\boldsymbol{R} \right\rangle ,\;
    \left| i\boldsymbol{R} \right\rangle 
      = \frac{V}{(2\pi)^3} \int \mathrm{d}\boldsymbol{k} e^{-i\boldsymbol{k} \cdot \boldsymbol{R}} 
        \left| i\boldsymbol{k} \right\rangle \\
    \begin{aligned}
      \text{Orthonormality:} \;&
        \left\langle i\boldsymbol{R} \middle| j\boldsymbol{R}' \right\rangle = \delta_{ij} \delta_{\boldsymbol{R}\boldsymbol{R}'} \\
      \text{Completeness:} \;&
        \hat{\mathcal{I}} = \sum_{i\boldsymbol{R}} \left| i\boldsymbol{R} \right\rangle \left\langle i\boldsymbol{R} \right| 
    \end{aligned}
  \end{gathered}
\end{equation}

\subsection{Bloch function $\left| n\boldsymbol{k} \right\rangle$}

In the low-energy subspace described by basis $\{\left| n\boldsymbol{k} \right\rangle | n \in \mathcal{G}\}$, 
we can get the numerical vector form of Bloch function $\left| n\boldsymbol{k} \right\rangle$ with wannier basis.

\begin{equation}
  \left| n \boldsymbol{k} \right\rangle 
    = \sum_{i} U_{in}^{\boldsymbol{k}} \left| i\boldsymbol{k} \right\rangle 
    = \sum_{i\boldsymbol{R}} U_{in}^{\boldsymbol{k}} e^{i\boldsymbol{k} \cdot \boldsymbol{R}} \left| i\boldsymbol{R} \right\rangle 
\end{equation}

\noindent
With $\hat{H} \left| n \boldsymbol{k} \right\rangle = E_{n \boldsymbol{k}} \left| n \boldsymbol{k} \right\rangle$, we can get

\begin{equation}
  \begin{aligned}
    \left\langle i \boldsymbol{k}' \right| H \left| n \boldsymbol{k} \right\rangle 
      &= \sum_{j} U_{jn}^{\boldsymbol{k}} \sum_{\boldsymbol{R}'\boldsymbol{R}} 
        e^{i\left( \boldsymbol{k} \cdot \boldsymbol{R} - \boldsymbol{k}' \cdot \boldsymbol{R}' \right)} 
        \left\langle i \boldsymbol{R}' \right| H \left| j \boldsymbol{R} \right\rangle \\
      &= \sum_{j} U_{jn}^{\boldsymbol{k}} \sum_{\boldsymbol{R}'\boldsymbol{R}} 
        e^{i\left( \boldsymbol{k} - \boldsymbol{k}' \right) \cdot \boldsymbol{R}'} 
        e^{i\boldsymbol{k} \cdot \left( \boldsymbol{R} - \boldsymbol{R}' \right)} 
        \left\langle i \boldsymbol{0} \right| H \left| j \boldsymbol{R} - \boldsymbol{R}' \right\rangle \\
      &= \sum_{j} U_{jn}^{\boldsymbol{k}} \sum_{\boldsymbol{R}_0} e^{i\left( \boldsymbol{k} - \boldsymbol{k}' \right) \cdot \boldsymbol{R}_0} 
        \sum_{\boldsymbol{R}} e^{i\boldsymbol{k} \cdot \boldsymbol{R}} 
        \left\langle i \boldsymbol{0} \right| H \left| j \boldsymbol{R} \right\rangle \\
      &= \frac{(2\pi)^3}{V} \delta(\boldsymbol{k}-\boldsymbol{k}') \sum_{j} H_{ij}^{\boldsymbol{k}} U_{jn}^{\boldsymbol{k}} \\
    \left\langle i \boldsymbol{k}' \right| E_{n\boldsymbol{k}} \left| n \boldsymbol{k} \right\rangle 
      &= \frac{(2\pi)^3}{V} \delta(\boldsymbol{k}-\boldsymbol{k}') E_{n\boldsymbol{k}} U_{in}^{\boldsymbol{k}} 
  \end{aligned}
\end{equation}

\begin{equation}
  \Rightarrow \sum_{j} H_{ij}^{\boldsymbol{k}} U_{jn}^{\boldsymbol{k}} = E_{n\boldsymbol{k}} U_{in}^{\boldsymbol{k}} 
\end{equation}

\noindent
where

\begin{equation}
  H_{ij}^{\boldsymbol{k}} = \sum_{\boldsymbol{R}} e^{i\boldsymbol{k} \cdot \boldsymbol{R}} \left\langle i \boldsymbol{0} \right| H \left| j \boldsymbol{R} \right\rangle
\end{equation}

\subsection{$\left| u_{n\boldsymbol{k}} \right\rangle = e^{-i\boldsymbol{k} \cdot \hat{\boldsymbol{r}}} \left| n\boldsymbol{k} \right\rangle$}

Now we try to get $\left| u_{n\boldsymbol{k}} \right\rangle$, the periodic part of Bloch function $\left| n\boldsymbol{k} \right\rangle$. 

\begin{equation}
  \begin{gathered}
    \left| u_{n\boldsymbol{k}} \right\rangle = e^{-i\boldsymbol{k} \cdot \hat{\boldsymbol{r}}} \left| n\boldsymbol{k} \right\rangle 
      = \sum_{i} U_{in}^{\boldsymbol{k}} e^{-i\boldsymbol{k} \cdot \hat{\boldsymbol{r}}} \left| i\boldsymbol{k} \right\rangle 
      = \sum_{i} U_{in}^{\boldsymbol{k}} \left| u_{i\boldsymbol{k}} \right\rangle \\
    \left| u_{i\boldsymbol{k}} \right\rangle 
      = \sum_{\boldsymbol{R}} e^{i\boldsymbol{k} \cdot (\boldsymbol{R} - \hat{\boldsymbol{r}})} \left| i\boldsymbol{R} \right\rangle 
  \end{gathered}
\end{equation}

Whether in the field of physical theories or mathematical theories, 
we only need the unitcell's inner product of $\left| u_{n\boldsymbol{k}} \right\rangle$ between the adjacent $\boldsymbol{k}$, 
which can be used to calculate physical quantities such as the Berry connection.
So we have

\begin{equation}
  \begin{aligned}
    \lim_{\boldsymbol{k} \rightarrow \boldsymbol{k}'} \left\langle u_{i\boldsymbol{k}'} \middle| u_{j\boldsymbol{k}} \right\rangle_{uc} 
      &= \frac{1}{\sum_{\boldsymbol{R}_0}} \left\langle u_{i\boldsymbol{k}'} \middle| u_{j\boldsymbol{k}} \right\rangle \\
      &= \frac{1}{\sum_{\boldsymbol{R}_0}} \sum_{\boldsymbol{R}'\boldsymbol{R}} 
        e^{i\left( \boldsymbol{k} \cdot \boldsymbol{R} - \boldsymbol{k}' \cdot \boldsymbol{R}' \right)} 
        \left\langle i\boldsymbol{R}' \right| e^{i(\boldsymbol{k}'-\boldsymbol{k}) \cdot \hat{\boldsymbol{r}}} \left| j\boldsymbol{R} \right\rangle \\
      &= \frac{1}{\sum_{\boldsymbol{R}_0}} \sum_{\boldsymbol{R}_0\boldsymbol{R}} 
        e^{i\boldsymbol{k} \cdot \boldsymbol{R}} 
        \left\langle i\boldsymbol{0} \right| e^{i(\boldsymbol{k}'-\boldsymbol{k}) \cdot \hat{\boldsymbol{r}}} \left| j\boldsymbol{R} \right\rangle \\
      &= \sum_{\boldsymbol{R}} e^{i\boldsymbol{k} \cdot \boldsymbol{R}} 
        \left\langle i\boldsymbol{0} \right| \left[ 1 + i(\boldsymbol{k}'-\boldsymbol{k}) \cdot \hat{\boldsymbol{r}} \right] \left| j\boldsymbol{R} \right\rangle
  \end{aligned}
\end{equation}

If $\left\langle \boldsymbol{r} \middle| i\boldsymbol{R} \right\rangle$ is the Maximal Localized Wannier Function(MLWF), 
$\left| i\boldsymbol{R} \right\rangle$ will be the eigen state of projected position operator:

\begin{equation}
  \hat{\mathcal{P}} = \frac{V}{(2\pi)^3} \int \mathrm{d}\boldsymbol{k} \sum_{n}^{\mathcal{G}} \left| n\boldsymbol{k} \right\rangle \left\langle n\boldsymbol{k} \right| 
\end{equation}

\begin{equation}
  \hat{\mathcal{P}} \hat{\boldsymbol{r}} \hat{\mathcal{P}} \left| i\boldsymbol{R} \right\rangle 
    = \left( \boldsymbol{R} + \boldsymbol{\tau}_{i} \right) \left| i\boldsymbol{R} \right\rangle 
\end{equation}

\noindent
then

\begin{equation}
  \begin{aligned}
    \lim_{\boldsymbol{k} \rightarrow \boldsymbol{k}'} \left\langle u_{i\boldsymbol{k}'} \middle| u_{j\boldsymbol{k}} \right\rangle_{uc} 
    &= \sum_{\boldsymbol{R}} e^{i\boldsymbol{k} \cdot \boldsymbol{R}} 
        \left\langle i\boldsymbol{0} \right| \hat{\mathcal{P}} 
        \left[ 1 + i(\boldsymbol{k}'-\boldsymbol{k}) \cdot \hat{\boldsymbol{r}} \right] 
        \hat{\mathcal{P}} \left| j\boldsymbol{R} \right\rangle \\
    &= \sum_{\boldsymbol{R}} e^{i\boldsymbol{k} \cdot \boldsymbol{R}} 
      \left[ 1 + i(\boldsymbol{k}'-\boldsymbol{k}) \cdot \boldsymbol{\tau}_{i} \right] 
      \delta_{ij}\delta_{\boldsymbol{R}\boldsymbol{0}} \\
    &= \delta_{ij} \left[ 1 + i(\boldsymbol{k}'-\boldsymbol{k}) \cdot \boldsymbol{\tau}_{i} \right] \\
    &= \delta_{ij} e^{i(\boldsymbol{k}'-\boldsymbol{k}) \cdot \boldsymbol{\tau}_{i}} 
  \end{aligned}
\end{equation}

\noindent
For the band state,

\begin{equation}
  \begin{aligned}
    \lim_{\boldsymbol{k} \rightarrow \boldsymbol{k}'} \left\langle u_{n'\boldsymbol{k}'} \middle| u_{n\boldsymbol{k}} \right\rangle_{uc} 
    &= \sum_{i} e^{i(\boldsymbol{k}'-\boldsymbol{k}) \cdot \boldsymbol{\tau}_{i}} U_{in'}^{\boldsymbol{k}'*} U_{in}^{\boldsymbol{k}}
  \end{aligned}
\end{equation}

So we can use ${U'}_{in}^{\boldsymbol{k}} = e^{-i\boldsymbol{k} \cdot \boldsymbol{\tau}_{i}} U_{in}^{\boldsymbol{k}}$ to represent $\left| u_{nk} \right\rangle$,
their scalar product give the same result

\begin{equation}
  \sum_{i} {U'}_{in'}^{\boldsymbol{k}'*} {U'}_{in}^{\boldsymbol{k}} 
    = \sum_{i} e^{i(\boldsymbol{k}'-\boldsymbol{k}) \cdot \boldsymbol{\tau}_{i}} U_{in'}^{\boldsymbol{k}'*} U_{in}^{\boldsymbol{k}}
\end{equation}

\noindent
Acctually, the ${U'}_{in}^{\boldsymbol{k}}$ is the eigen vector of Hamiltonian under atomic guage.

\subsection{Finite case}

Our above derivation we presented is for the case where the system is infinite.
In this section, we will give some definitions in the case where the system is finite with periodic boundary condition.

\noindent
For Bloch states:

\begin{equation}
  \begin{aligned}
    \text{Orthonormality:} \;&
    \left\langle m\boldsymbol{k} \middle| n\boldsymbol{k}' \right\rangle 
      = \delta_{mn} \delta_{\boldsymbol{k}\boldsymbol{k}'} \\
    \text{Completeness:} \;&
    \hat{\mathcal{I}} = \sum_{n\boldsymbol{k}} \left| n\boldsymbol{k} \right\rangle \left\langle n\boldsymbol{k} \right| 
  \end{aligned}
\end{equation}

\noindent
For Wannier states:

\begin{equation}
  \begin{gathered}
    \left| i\boldsymbol{k} \right\rangle 
      = \frac{1}{\sqrt{N}} \sum_{\boldsymbol{R}} e^{i\boldsymbol{k} \cdot \boldsymbol{R}} \left| i\boldsymbol{R} \right\rangle ,\;
    \left| i\boldsymbol{R} \right\rangle 
      = \frac{1}{\sqrt{N}} \sum_{\boldsymbol{k}} e^{-i\boldsymbol{k} \cdot \boldsymbol{R}} 
        \left| i\boldsymbol{k} \right\rangle \\
    \begin{aligned}
      \text{Orthonormality:} \;&
        \left\langle i\boldsymbol{R} \middle| j\boldsymbol{R}' \right\rangle = \delta_{ij} \delta_{\boldsymbol{R}\boldsymbol{R}'} \\
      \text{Completeness:} \;&
        \hat{\mathcal{I}} = \sum_{i\boldsymbol{R}} \left| i\boldsymbol{R} \right\rangle \left\langle i\boldsymbol{R} \right| 
    \end{aligned}
  \end{gathered}
\end{equation}

\section{Topology}



\newpage

\chapter{Excitonic BSE Model}

\section{BSE based on electronic wannier basis}

The standard Bethe-Salpeter equation for exciton is

\begin{equation}
  (E_{c\boldsymbol{k} + \boldsymbol{q}} - E_{v\boldsymbol{k}}) A_{vc}^{\alpha}(\boldsymbol{q},\boldsymbol{k}) 
  + \frac{V}{(2\pi)^3} \int \mathrm{d}\boldsymbol{k}' \sum_{v'c'} \left[
    K_{vc,v'c'}^{d}(\boldsymbol{q},\boldsymbol{k},\boldsymbol{k}') + K_{vc,v'c'}^{x} (\boldsymbol{q},\boldsymbol{k},\boldsymbol{k}')
  \right] A_{v'c'}^{\alpha} (\boldsymbol{q},\boldsymbol{k}') 
  = E_{\alpha \boldsymbol{q}} A_{vc}^{\alpha}(\boldsymbol{q},\boldsymbol{k}) 
\end{equation}

\noindent
where

\begin{equation}
  \begin{aligned}
    K_{v'c'\boldsymbol{k}'vc\boldsymbol{k}}^{d\,\boldsymbol{q}} 
      &= -\left\langle c'\boldsymbol{k}'+\boldsymbol{q};v\boldsymbol{k} \left| W \right| v'\boldsymbol{k}';c\boldsymbol{k}+\boldsymbol{q} \right\rangle \\
    K_{v'c'\boldsymbol{k}'vc\boldsymbol{k}}^{x\,\boldsymbol{q}} 
      &= \left\langle c'\boldsymbol{k}'+\boldsymbol{q};v\boldsymbol{k} \left| V \right| c\boldsymbol{k}+\boldsymbol{q};v'\boldsymbol{k}' \right\rangle
  \end{aligned}
\end{equation}

\noindent
Note here we define

\begin{equation}
  \left\langle i;j \left| F \right| k;l \right\rangle 
    = \iint \mathrm{d} \boldsymbol{r} \mathrm{d} \boldsymbol{r}'
      \psi_{i}^{*}(\boldsymbol{r}) \psi_{j}^{*}(\boldsymbol{r}') 
      F(\boldsymbol{r},\boldsymbol{r}') 
      \psi_{k}^{}(\boldsymbol{r}') \psi_{l}^{}(\boldsymbol{r})
\end{equation}

Thretically, we should imagine a infinite system and $\boldsymbol{k}$ should be a continuous variable. 
But the BSE is mainly used to do practical calculation, which prefer a discrete form.
Note a discrete grid of kpoints can be regarded as sampling grids for numerically calculating continuous integrals.
This sampling perspective can help us understand the subsequent processing steps.

With electronic wannier functions, we can expand the Kernal terms as

\begin{equation}
  \begin{aligned}
    K_{v'c'\boldsymbol{k}'vc\boldsymbol{k}}^{d\,\boldsymbol{q}} 
      &= - \sum_{ijkl} U_{ic'}^{\boldsymbol{k}'+\boldsymbol{q}*} U_{jv}^{\boldsymbol{k}*} 
        U_{kv'}^{\boldsymbol{k}'} U_{lc}^{\boldsymbol{k}+\boldsymbol{q}} 
        \left\langle i\boldsymbol{k}'+\boldsymbol{q};j\boldsymbol{k} \left| W \right| k\boldsymbol{k}';l\boldsymbol{k}+\boldsymbol{q} \right\rangle \\
      &= - \sum_{ijkl} U_{ic'}^{\boldsymbol{k}'+\boldsymbol{q}*} U_{jv}^{\boldsymbol{k}*} 
        U_{kv'}^{\boldsymbol{k}'} U_{lc}^{\boldsymbol{k}+\boldsymbol{q}} 
        \frac{1}{N^2} \sum_{\{\boldsymbol{R}\}} 
        e^{-i(\boldsymbol{k}'+\boldsymbol{q}) \cdot \boldsymbol{R}_1} 
        e^{-i\boldsymbol{k} \cdot \boldsymbol{R}_2} 
        e^{i\boldsymbol{k}' \cdot \boldsymbol{R}_3} 
        e^{i(\boldsymbol{k}+\boldsymbol{q}) \cdot \boldsymbol{R}_4} 
        \left\langle i\boldsymbol{R}_1;j\boldsymbol{R}_2 \left| W \right| k\boldsymbol{R}_3;l\boldsymbol{R}_4 \right\rangle \\
      &= - \sum_{ijkl} U_{ic'}^{\boldsymbol{k}'+\boldsymbol{q}*} U_{jv}^{\boldsymbol{k}*} U_{kv'}^{\boldsymbol{k}'} U_{lc}^{\boldsymbol{k}+\boldsymbol{q}} 
        \frac{1}{N^2} \sum_{\{\boldsymbol{R}\}} 
        e^{-i(\boldsymbol{k}'+\boldsymbol{q}) \cdot \boldsymbol{R}_1} 
        e^{-i\boldsymbol{k} \cdot \boldsymbol{R}_2} 
        e^{i\boldsymbol{k}' \cdot \boldsymbol{R}_3} 
        e^{i(\boldsymbol{k}+\boldsymbol{q}) \cdot \boldsymbol{R}_4} 
        W_{i\boldsymbol{R}_1,j\boldsymbol{R}_2,k\boldsymbol{R}_3,l\boldsymbol{R}_4} \\
      &= - \sum_{ijkl} U_{ic'}^{\boldsymbol{k}'+\boldsymbol{q}*} U_{jv}^{\boldsymbol{k}*} U_{kv'}^{\boldsymbol{k}'} U_{lc}^{\boldsymbol{k}+\boldsymbol{q}} 
        \frac{1}{N} \sum_{\boldsymbol{R}_1\boldsymbol{R}_2\boldsymbol{R}_3} 
        e^{-i\boldsymbol{k} \cdot \boldsymbol{R}_1} 
        e^{i\boldsymbol{k}' \cdot \boldsymbol{R}_2} 
        e^{i(\boldsymbol{k}+\boldsymbol{q}) \cdot \boldsymbol{R}_3} 
        W_{i\boldsymbol{0},j\boldsymbol{R}_1,k\boldsymbol{R}_2,l\boldsymbol{R}_3} \\
  \end{aligned}
\end{equation}

\begin{equation}
  \begin{aligned}
    K_{v'c'\boldsymbol{k}'vc\boldsymbol{k}}^{x\,\boldsymbol{q}} 
      &= \sum_{ijkl} U_{ic'}^{\boldsymbol{k}'+\boldsymbol{q}*} U_{jv}^{\boldsymbol{k}*} 
        U_{kc}^{\boldsymbol{k}+\boldsymbol{q}} U_{lv'}^{\boldsymbol{k}'} 
        \left\langle i\boldsymbol{k}'+\boldsymbol{q};j\boldsymbol{k} \left| V \right| k\boldsymbol{k}+\boldsymbol{q};l\boldsymbol{k}' \right\rangle \\
      &= \sum_{ijkl} U_{ic'}^{\boldsymbol{k}'+\boldsymbol{q}*} U_{jv}^{\boldsymbol{k}*} 
        U_{kc}^{\boldsymbol{k}+\boldsymbol{q}} U_{lv'}^{\boldsymbol{k}'} 
        \frac{1}{N^2} \sum_{\{\boldsymbol{R}\}} 
        e^{-i(\boldsymbol{k}'+\boldsymbol{q}) \cdot \boldsymbol{R}_1} 
        e^{-i\boldsymbol{k} \cdot \boldsymbol{R}_2} 
        e^{i(\boldsymbol{k}+\boldsymbol{q}) \cdot \boldsymbol{R}_3} 
        e^{i\boldsymbol{k}' \cdot \boldsymbol{R}_4} 
        \left\langle i\boldsymbol{R}_1;j\boldsymbol{R}_2 \left| V \right| k\boldsymbol{R}_3;l\boldsymbol{R}_4 \right\rangle \\
      &= \sum_{ijkl} U_{ic'}^{\boldsymbol{k}'+\boldsymbol{q}*} U_{jv}^{\boldsymbol{k}*} U_{kv'}^{\boldsymbol{k}'} U_{lc}^{\boldsymbol{k}+\boldsymbol{q}} 
        \frac{1}{N^2} \sum_{\{\boldsymbol{R}\}} 
        e^{-i(\boldsymbol{k}'+\boldsymbol{q}) \cdot \boldsymbol{R}_1} 
        e^{-i\boldsymbol{k} \cdot \boldsymbol{R}_2} 
        e^{i(\boldsymbol{k}+\boldsymbol{q}) \cdot \boldsymbol{R}_3} 
        e^{i\boldsymbol{k}' \cdot \boldsymbol{R}_4} 
        V_{i\boldsymbol{R}_1,j\boldsymbol{R}_2,k\boldsymbol{R}_3,l\boldsymbol{R}_4} \\
      &= \sum_{ijkl} U_{ic'}^{\boldsymbol{k}'+\boldsymbol{q}*} U_{jv}^{\boldsymbol{k}*} U_{kv'}^{\boldsymbol{k}'} U_{lc}^{\boldsymbol{k}+\boldsymbol{q}} 
        \frac{1}{N} \sum_{\boldsymbol{R}_1\boldsymbol{R}_2\boldsymbol{R}_3} 
        e^{-i\boldsymbol{k} \cdot \boldsymbol{R}_1} 
        e^{i(\boldsymbol{k}+\boldsymbol{q}) \cdot \boldsymbol{R}_2}
        e^{i\boldsymbol{k}' \cdot \boldsymbol{R}_3} 
        V_{i\boldsymbol{0},j\boldsymbol{R}_1,k\boldsymbol{R}_2,l\boldsymbol{R}_3} \\
  \end{aligned}
\end{equation}

\subsection{Kernal under UJ approximation}

If we only keep the direct term $U$ and the exchange term $J$, we can simplify the Kernal to




\section{Excitonic Band state}



\section{Excitonic Topology}

\chapter{Interaction parameters between wannier functions}

\section{Direct terms}

\subsection{Mirror Correction}

\subsection{Long-range Correction}

\section{The practical calculation of Interaction parameters}

\chapter{Appendix}


\section{Gaussian potential}

\bibliographystyle{unsrt}
%输出参考文献
\bibliography{bib/database}
\nocite{*} %显示数据库中有的,但是正文没有引用的文献

\end{document}
