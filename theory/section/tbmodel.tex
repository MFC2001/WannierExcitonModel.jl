\section{Tight Binding Model}

We assume that reader have the basic knowledge of "Band Theory" and "Tight-Binding Model",
So this chaper is mainly used to define the relevant notations.

\subsection{Poison summation}

\noindent
Finite case:

\begin{equation}
  \begin{aligned}
    \sum_{\boldsymbol{R}} e^{i(\boldsymbol{k}-\boldsymbol{k}') \cdot \boldsymbol{R}} &= N \delta_{\boldsymbol{k}\boldsymbol{k}'} \\
    \sum_{\boldsymbol{k}} e^{i\boldsymbol{k} \cdot (\boldsymbol{R}-\boldsymbol{R}')} &= N \delta_{\boldsymbol{R}\boldsymbol{R}'} 
  \end{aligned}
\end{equation}

\noindent
Infinite case:

\begin{equation}
  \begin{aligned}
    \sum_{\boldsymbol{R}} e^{i(\boldsymbol{k}-\boldsymbol{k}') \cdot \boldsymbol{R}} = \frac{(2\pi)^3}{V} \delta(\boldsymbol{k}-\boldsymbol{k}') \\
    \frac{V}{(2\pi)^3} \int \mathrm{d}\boldsymbol{k} e^{i\boldsymbol{k} \cdot (\boldsymbol{R}-\boldsymbol{R}')} = \delta_{\boldsymbol{R}\boldsymbol{R}'} 
  \end{aligned}
\end{equation}


\subsection{Band state}

For the case where the system is infinite, we define

\begin{equation}
  \begin{aligned}
    \text{Orthonormality:} \;&
    \left\langle m\boldsymbol{k} \middle| n\boldsymbol{k}' \right\rangle 
      = \frac{(2\pi)^3}{V} \delta_{mn} \delta(\boldsymbol{k}-\boldsymbol{k}') \\
    \text{Completeness:} \;&
    \hat{\mathcal{I}} = \frac{V}{(2\pi)^3} \int \mathrm{d}\boldsymbol{k} 
      \sum_{n} \left| n\boldsymbol{k} \right\rangle \left\langle n\boldsymbol{k} \right| 
  \end{aligned}
\end{equation}

\noindent
where $\left| n\boldsymbol{k} \right\rangle$ is the single partical Bloch state.
Under unitary transformation $U_{in}^{\boldsymbol{k}}$, these two equations maintain the formation:

\begin{equation}
  \begin{gathered}
    \left| n \boldsymbol{k} \right\rangle 
      = \sum_{i} U_{in}^{\boldsymbol{k}} \left| i\boldsymbol{k} \right\rangle ,\;
    \left| i\boldsymbol{k} \right\rangle = \sum_{n} U_{in}^{\boldsymbol{k}*} \left| n \boldsymbol{k} \right\rangle \\
    \begin{aligned}
      \text{Orthonormality:} \;&
        \left\langle i\boldsymbol{k} \middle| j\boldsymbol{k}' \right\rangle = \frac{(2\pi)^3}{V} \delta_{ij} \delta(\boldsymbol{k}-\boldsymbol{k}') \\
      \text{Completeness:} \;&
        \hat{\mathcal{I}} = \frac{V}{(2\pi)^3} \int \mathrm{d}\boldsymbol{k} \sum_{i} \left| i\boldsymbol{k} \right\rangle \left\langle i\boldsymbol{k} \right| 
    \end{aligned}
  \end{gathered}
\end{equation}

\noindent
We can define wannier states from Bloch states, and deduce its orthonormality and completeness:

\begin{equation}
  \begin{gathered}
    \left| i\boldsymbol{k} \right\rangle 
      = \sum_{\boldsymbol{R}} e^{i\boldsymbol{k} \cdot \boldsymbol{R}} \left| i\boldsymbol{R} \right\rangle ,\;
    \left| i\boldsymbol{R} \right\rangle 
      = \frac{V}{(2\pi)^3} \int \mathrm{d}\boldsymbol{k} e^{-i\boldsymbol{k} \cdot \boldsymbol{R}} 
        \left| i\boldsymbol{k} \right\rangle \\
    \begin{aligned}
      \text{Orthonormality:} \;&
        \left\langle i\boldsymbol{R} \middle| j\boldsymbol{R}' \right\rangle = \delta_{ij} \delta_{\boldsymbol{R}\boldsymbol{R}'} \\
      \text{Completeness:} \;&
        \hat{\mathcal{I}} = \sum_{i\boldsymbol{R}} \left| i\boldsymbol{R} \right\rangle \left\langle i\boldsymbol{R} \right| 
    \end{aligned}
  \end{gathered}
\end{equation}

\subsubsection{Bloch state $\left| n\boldsymbol{k} \right\rangle$}

We can get the numerical vector form of Bloch function $\left| n\boldsymbol{k} \right\rangle$ with wannier basis.

\begin{equation}
  \left| n \boldsymbol{k} \right\rangle 
    = \sum_{i} U_{in}^{\boldsymbol{k}} \left| i\boldsymbol{k} \right\rangle 
    = \sum_{i\boldsymbol{R}} U_{in}^{\boldsymbol{k}} e^{i\boldsymbol{k} \cdot \boldsymbol{R}} \left| i\boldsymbol{R} \right\rangle 
\end{equation}

\noindent
With $\hat{H} \left| n \boldsymbol{k} \right\rangle = E_{n \boldsymbol{k}} \left| n \boldsymbol{k} \right\rangle$, we can get

\begin{equation*}
  \begin{aligned}
    \left\langle i \boldsymbol{k}' \right| H \left| n \boldsymbol{k} \right\rangle 
      &= \sum_{j} U_{jn}^{\boldsymbol{k}} \sum_{\boldsymbol{R}'\boldsymbol{R}} 
        e^{i\left( \boldsymbol{k} \cdot \boldsymbol{R} - \boldsymbol{k}' \cdot \boldsymbol{R}' \right)} 
        \left\langle i \boldsymbol{R}' \right| H \left| j \boldsymbol{R} \right\rangle \\
      &= \sum_{j} U_{jn}^{\boldsymbol{k}} \sum_{\boldsymbol{R}'\boldsymbol{R}} 
        e^{i\left( \boldsymbol{k} - \boldsymbol{k}' \right) \cdot \boldsymbol{R}'} 
        e^{i\boldsymbol{k} \cdot \left( \boldsymbol{R} - \boldsymbol{R}' \right)} 
        \left\langle i \boldsymbol{0} \right| H \left| j \boldsymbol{R} - \boldsymbol{R}' \right\rangle \\
      &= \sum_{j} U_{jn}^{\boldsymbol{k}} \sum_{\boldsymbol{R}_0} e^{i\left( \boldsymbol{k} - \boldsymbol{k}' \right) \cdot \boldsymbol{R}_0} 
        \sum_{\boldsymbol{R}} e^{i\boldsymbol{k} \cdot \boldsymbol{R}} 
        \left\langle i \boldsymbol{0} \right| H \left| j \boldsymbol{R} \right\rangle \\
      &= \frac{(2\pi)^3}{V} \delta(\boldsymbol{k}-\boldsymbol{k}') \sum_{j} H_{ij}^{\boldsymbol{k}} U_{jn}^{\boldsymbol{k}} \\
    \left\langle i \boldsymbol{k}' \right| E_{n\boldsymbol{k}} \left| n \boldsymbol{k} \right\rangle 
      &= \frac{(2\pi)^3}{V} \delta(\boldsymbol{k}-\boldsymbol{k}') E_{n\boldsymbol{k}} U_{in}^{\boldsymbol{k}} 
  \end{aligned}
\end{equation*}

\begin{equation}
  \Rightarrow \sum_{j} H_{ij}^{\boldsymbol{k}} U_{jn}^{\boldsymbol{k}} = E_{n\boldsymbol{k}} U_{in}^{\boldsymbol{k}} 
\end{equation}

\noindent
where

\begin{equation}
  H_{ij}^{\boldsymbol{k}} = \sum_{\boldsymbol{R}} e^{i\boldsymbol{k} \cdot \boldsymbol{R}} \left\langle i \boldsymbol{0} \right| H \left| j \boldsymbol{R} \right\rangle
\end{equation}

\subsubsection{$\left| u_{n\boldsymbol{k}} \right\rangle = e^{-i\boldsymbol{k} \cdot \hat{\boldsymbol{r}}} \left| n\boldsymbol{k} \right\rangle$}

Now we try to get $\left| u_{n\boldsymbol{k}} \right\rangle$, the periodic part of Bloch function $\left| n\boldsymbol{k} \right\rangle$. 

\begin{equation}
  \begin{gathered}
    \left| u_{n\boldsymbol{k}} \right\rangle = e^{-i\boldsymbol{k} \cdot \hat{\boldsymbol{r}}} \left| n\boldsymbol{k} \right\rangle 
      = \sum_{i} U_{in}^{\boldsymbol{k}} e^{-i\boldsymbol{k} \cdot \hat{\boldsymbol{r}}} \left| i\boldsymbol{k} \right\rangle 
      = \sum_{i} U_{in}^{\boldsymbol{k}} \left| u_{i\boldsymbol{k}} \right\rangle \\
    \left| u_{i\boldsymbol{k}} \right\rangle 
      = \sum_{\boldsymbol{R}} e^{i\boldsymbol{k} \cdot (\boldsymbol{R} - \hat{\boldsymbol{r}})} \left| i\boldsymbol{R} \right\rangle 
  \end{gathered}
\end{equation}

Whether in the field of physical theories or mathematical theories, 
we only need the unitcell's inner product of $\left| u_{n\boldsymbol{k}} \right\rangle$ between the adjacent $\boldsymbol{k}$, 
which can be used to calculate physical quantities such as the Berry connection.
So we have

\begin{equation}
  \begin{aligned}
    \lim_{\boldsymbol{k} \rightarrow \boldsymbol{k}'} \left\langle u_{i\boldsymbol{k}'} \middle| u_{j\boldsymbol{k}} \right\rangle_{uc} 
      &= \frac{1}{\sum_{\boldsymbol{R}}} \left\langle u_{i\boldsymbol{k}'} \middle| u_{j\boldsymbol{k}} \right\rangle \\
      &= \frac{1}{\sum_{\boldsymbol{R}}} \sum_{\boldsymbol{R}'\boldsymbol{R}} 
        e^{i\left( \boldsymbol{k} \cdot \boldsymbol{R} - \boldsymbol{k}' \cdot \boldsymbol{R}' \right)} 
        \left\langle i\boldsymbol{R}' \right| e^{i(\boldsymbol{k}'-\boldsymbol{k}) \cdot \hat{\boldsymbol{r}}} \left| j\boldsymbol{R} \right\rangle \\
      &= \frac{1}{\sum_{\boldsymbol{R}}} \sum_{\boldsymbol{R}_0\boldsymbol{R}} 
        e^{i\boldsymbol{k} \cdot \boldsymbol{R}} 
        \left\langle i\boldsymbol{0} \right| e^{i(\boldsymbol{k}'-\boldsymbol{k}) \cdot \hat{\boldsymbol{r}}} \left| j\boldsymbol{R} \right\rangle \\
      &= \sum_{\boldsymbol{R}} e^{i\boldsymbol{k} \cdot \boldsymbol{R}} 
        \left\langle i\boldsymbol{0} \right| \left[ 1 + i(\boldsymbol{k}'-\boldsymbol{k}) \cdot \hat{\boldsymbol{r}} \right] \left| j\boldsymbol{R} \right\rangle
  \end{aligned}
\end{equation}

\noindent
In a low-energy subspace described by basis $\{\left| n\boldsymbol{k} \right\rangle | n \in \mathcal{G}\}$, 
we define $\left\langle \boldsymbol{r} \middle| i\boldsymbol{R} \right\rangle$ as the Maximal Localized Wannier Function(MLWF).
If $\mathcal{G}$ represents the entire Hilbert space, MLWF is $\delta$-function, the eigen states of position operator.
Otherwise, under the localization criterion introduced by [Marzari and Vanderbilt(1997)], 
$\left| i\boldsymbol{R} \right\rangle$ will be the eigen state of projected position operator in the subspace:

\begin{equation}
  \hat{\mathcal{P}} 
  = \frac{V}{(2\pi)^3} \int \mathrm{d}\boldsymbol{k} \sum_{n}^{\mathcal{G}} \left| n\boldsymbol{k} \right\rangle \left\langle n\boldsymbol{k} \right| 
  = \sum_{i \boldsymbol{R}} \left| i\boldsymbol{R} \right\rangle \left\langle i\boldsymbol{R} \right| 
\end{equation}

\begin{equation}
  \hat{\mathcal{P}} \hat{\boldsymbol{r}} \hat{\mathcal{P}} \left| i\boldsymbol{R} \right\rangle 
    = \left( \boldsymbol{R} + \boldsymbol{\tau}_{i} \right) \left| i\boldsymbol{R} \right\rangle 
\end{equation}

\noindent
then

\begin{equation}
  \begin{aligned}
    \lim_{\boldsymbol{k} \rightarrow \boldsymbol{k}'} \left\langle u_{i\boldsymbol{k}'} \middle| u_{j\boldsymbol{k}} \right\rangle_{uc} 
    &= \sum_{\boldsymbol{R}} e^{i\boldsymbol{k} \cdot \boldsymbol{R}} 
        \left\langle i\boldsymbol{0} \right| \hat{\mathcal{P}} 
        \left[ 1 + i(\boldsymbol{k}'-\boldsymbol{k}) \cdot \hat{\boldsymbol{r}} \right] 
        \hat{\mathcal{P}} \left| j\boldsymbol{R} \right\rangle \\
    &= \sum_{\boldsymbol{R}} e^{i\boldsymbol{k} \cdot \boldsymbol{R}} 
      \left[ 1 + i(\boldsymbol{k}'-\boldsymbol{k}) \cdot \boldsymbol{\tau}_{i} \right] 
      \delta_{ij}\delta_{\boldsymbol{R}\boldsymbol{0}} \\
    &= \delta_{ij} \left[ 1 + i(\boldsymbol{k}'-\boldsymbol{k}) \cdot \boldsymbol{\tau}_{i} \right] \\
    &= \delta_{ij} e^{i(\boldsymbol{k}'-\boldsymbol{k}) \cdot \boldsymbol{\tau}_{i}} 
  \end{aligned}
\end{equation}


For the band state,

\begin{equation}
  \begin{aligned}
    \lim_{\boldsymbol{k} \rightarrow \boldsymbol{k}'} \left\langle u_{n'\boldsymbol{k}'} \middle| u_{n\boldsymbol{k}} \right\rangle_{uc} 
    &= \sum_{i} e^{i(\boldsymbol{k}'-\boldsymbol{k}) \cdot \boldsymbol{\tau}_{i}} U_{in'}^{\boldsymbol{k}'*} U_{in}^{\boldsymbol{k}}
  \end{aligned}
\end{equation}

\noindent
So we can use ${U'}_{in}^{\boldsymbol{k}} = e^{-i\boldsymbol{k} \cdot \boldsymbol{\tau}_{i}} U_{in}^{\boldsymbol{k}}$ to represent $\left| u_{nk} \right\rangle$,
their scalar product give the same result

\begin{equation}
  \sum_{i} {U'}_{in'}^{\boldsymbol{k}'*} {U'}_{in}^{\boldsymbol{k}} 
    = \sum_{i} e^{i(\boldsymbol{k}'-\boldsymbol{k}) \cdot \boldsymbol{\tau}_{i}} U_{in'}^{\boldsymbol{k}'*} U_{in}^{\boldsymbol{k}}
\end{equation}

\noindent
Acctually, the ${U'}_{in}^{\boldsymbol{k}}$ is the eigen vector of Hamiltonian under atomic guage.

\subsection{Finite case}

Our above derivation we presented is for the case where the system is infinite.
In this section, we will give some definitions in the case where the system is finite with periodic boundary condition.

\noindent
For Bloch states:

\begin{equation}
  \begin{aligned}
    \text{Orthonormality:} \;&
    \left\langle m\boldsymbol{k} \middle| n\boldsymbol{k}' \right\rangle 
      = \delta_{mn} \delta_{\boldsymbol{k}\boldsymbol{k}'} \\
    \text{Completeness:} \;&
    \hat{\mathcal{I}} = \sum_{n\boldsymbol{k}} \left| n\boldsymbol{k} \right\rangle \left\langle n\boldsymbol{k} \right| 
  \end{aligned}
\end{equation}

\noindent
For Wannier states:

\begin{equation}
  \begin{gathered}
    \left| i\boldsymbol{k} \right\rangle 
      = \frac{1}{\sqrt{N}} \sum_{\boldsymbol{R}} e^{i\boldsymbol{k} \cdot \boldsymbol{R}} \left| i\boldsymbol{R} \right\rangle ,\;
    \left| i\boldsymbol{R} \right\rangle 
      = \frac{1}{\sqrt{N}} \sum_{\boldsymbol{k}} e^{-i\boldsymbol{k} \cdot \boldsymbol{R}} 
        \left| i\boldsymbol{k} \right\rangle \\
    \begin{aligned}
      \text{Orthonormality:} \;&
        \left\langle i\boldsymbol{R} \middle| j\boldsymbol{R}' \right\rangle = \delta_{ij} \delta_{\boldsymbol{R}\boldsymbol{R}'} \\
      \text{Completeness:} \;&
        \hat{\mathcal{I}} = \sum_{i\boldsymbol{R}} \left| i\boldsymbol{R} \right\rangle \left\langle i\boldsymbol{R} \right| 
    \end{aligned}
  \end{gathered}
\end{equation}

\subsection{Topology}