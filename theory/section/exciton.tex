\section{Excitonic Model}

\subsection{BSE based on electronic wannier basis}\label{BSE-based-on-electronic-wannier-basis}

In this section, we will start from the Bethe-Salpeter equation(BSE) in a system of infinite size.
The infinite perspective can help us to understand the interaction clearly.

\subsubsection{Excitonic Bloch state}\label{Excitonic-Bloch-state}

Single excitonic bloch state can be expressed as

\begin{equation}
  \left| \alpha \boldsymbol{q} \right\rangle 
    = \frac{V}{(2\pi)^3} \int \mathrm{d}\boldsymbol{k} \sum_{vc} \phi_{vc}^{\alpha \boldsymbol{q}}(\boldsymbol{k}) 
      \left| \psi_{c \boldsymbol{k} + \boldsymbol{q}}^{e} \right\rangle \left| \psi_{v \boldsymbol{k}}^{h} \right\rangle 
    = \frac{V}{(2\pi)^3} \int \mathrm{d}\boldsymbol{k} \sum_{vc} \phi_{vc}^{\alpha \boldsymbol{q}}(\boldsymbol{k}) 
      \left| v c \boldsymbol{k}, \boldsymbol{q} \right\rangle 
\end{equation}

\noindent
It's easy to get

\begin{equation}
  \begin{gathered}
    \begin{aligned}
      \left\langle v c \boldsymbol{k}, \boldsymbol{q} \middle| v c \boldsymbol{k}', \boldsymbol{q}' \right\rangle
        &= \frac{(2\pi)^6}{V^2} \delta_{vv'} \delta_{cc'} \delta(\boldsymbol{k}-\boldsymbol{k}') \delta(\boldsymbol{q}-\boldsymbol{q}') \\
      \left\langle \alpha \boldsymbol{q} \middle| \beta \boldsymbol{q}' \right\rangle
        &= \frac{(2\pi)^3}{V} \delta_{\alpha\beta} \delta(\boldsymbol{q}-\boldsymbol{q}') 
    \end{aligned} \\
    \frac{V}{(2\pi)^3} \int \mathrm{d}\boldsymbol{k} \sum_{vc} 
      \phi_{vc}^{\alpha \boldsymbol{q}*}(\boldsymbol{k}) \phi_{vc}^{\beta \boldsymbol{q}}(\boldsymbol{k}) 
      = \delta_{\alpha\beta}
  \end{gathered}
\end{equation}

\subsubsection{BSE}

Under Tamm-Dancoff approximation (TDA), the Bethe-Salpeter equation(BSE) for exciton in a infinite system is

\begin{equation}\label{BSE-infinite}
  (E_{c\boldsymbol{k} + \boldsymbol{q}} - E_{v\boldsymbol{k}}) \phi_{vc}^{\alpha\boldsymbol{q}}(\boldsymbol{k}) 
  + \frac{V^2}{(2\pi)^6} \iint \mathrm{d}\boldsymbol{q}' \mathrm{d}\boldsymbol{k}' \sum_{v'c'} 
    K_{vc,v'c'}^{}(\boldsymbol{q}, \boldsymbol{k}, \boldsymbol{q}', \boldsymbol{k}') 
    \phi_{v'c'}^{\alpha\boldsymbol{q}'} (\boldsymbol{k}') 
  = E_{\alpha \boldsymbol{q}} \phi_{vc}^{\alpha\boldsymbol{q}}(\boldsymbol{k}) 
\end{equation}

\noindent
where

\begin{equation}
  \begin{aligned}
    K_{vc,v'c'}^{} 
      &= K_{vcv'c'}^{d} + K_{vcv'c'}^{x} \\
    K_{vcv'c'}^{d}(\boldsymbol{q}, \boldsymbol{k}, \boldsymbol{q}', \boldsymbol{k}') 
      &= -\left\langle c\boldsymbol{k}+\boldsymbol{q};v'\boldsymbol{k}' \left| W \right| v\boldsymbol{k};c'\boldsymbol{k}'+\boldsymbol{q}' \right\rangle \\
    K_{vcv'c'}^{x}(\boldsymbol{q}, \boldsymbol{k}, \boldsymbol{q}', \boldsymbol{k}')
      &= \left\langle c\boldsymbol{k}+\boldsymbol{q};v'\boldsymbol{k}' \left| V \right| c'\boldsymbol{k}'+\boldsymbol{q}';v\boldsymbol{k} \right\rangle
  \end{aligned}
\end{equation}

\noindent
Here we define

\begin{equation}
  \left\langle i;j \left| F \right| k;l \right\rangle 
    = \iint \mathrm{d} \boldsymbol{r} \mathrm{d} \boldsymbol{r}'
      \psi_{i}^{*}(\boldsymbol{r}) \psi_{j}^{*}(\boldsymbol{r}') 
      F(\boldsymbol{r},\boldsymbol{r}') 
      \psi_{k}^{}(\boldsymbol{r}') \psi_{l}^{}(\boldsymbol{r})
\end{equation}

\noindent
Obviously we have $K_{vc,v'c'}^{} \propto \frac{(2\pi)^3}{V} \delta(\boldsymbol{q} - \boldsymbol{q}')$, 
which is consistent with the conservation of excitonic momentum.

In a finite system with periodic boundary condition, \cref{BSE-infinite} simply becomes

\begin{equation}\label{BSE-finite}
  (E_{c\boldsymbol{k} + \boldsymbol{q}} - E_{v\boldsymbol{k}}) A_{vc\boldsymbol{k}}^{\alpha\boldsymbol{q}} 
  + \sum_{v'c'\boldsymbol{k}'} K_{vc\boldsymbol{k},v'c'\boldsymbol{k}'}^{\boldsymbol{q}}
    A_{v'c'\boldsymbol{k}'}^{\alpha\boldsymbol{q}}
  = E_{\alpha \boldsymbol{q}} A_{vc\boldsymbol{k}}^{\alpha\boldsymbol{q}} 
\end{equation}

\noindent
This is used for practical calculation. 
We can catch the details by replacing the definition of electronic wave function and the substitution of summation and integration.
With electronic wannier functions, we can see the explicit process.


We can expand the Kernal terms with electronic wannier functions:

\begin{equation*}
  \begin{aligned}
    K_{vcv'c'}^{d}(\boldsymbol{q}, \boldsymbol{k}, \boldsymbol{q}', \boldsymbol{k}') 
      &= - \sum_{ijkl} U_{ic}^{\boldsymbol{k}+\boldsymbol{q}*} U_{jv'}^{\boldsymbol{k}'*} 
        U_{kv}^{\boldsymbol{k}} U_{lc'}^{\boldsymbol{k}'+\boldsymbol{q}'} 
        \left\langle i\boldsymbol{k}+\boldsymbol{q};j\boldsymbol{k}' \left| W \right| k\boldsymbol{k};l\boldsymbol{k}'+\boldsymbol{q}' \right\rangle \\
      &= - \sum_{ijkl} U_{ic}^{\boldsymbol{k}+\boldsymbol{q}*} U_{jv'}^{\boldsymbol{k}'*} 
        U_{kv}^{\boldsymbol{k}} U_{lc'}^{\boldsymbol{k}'+\boldsymbol{q}'} 
        \sum_{\{\boldsymbol{R}\}} 
        e^{-i(\boldsymbol{k}+\boldsymbol{q}) \cdot \boldsymbol{R}_1} 
        e^{-i\boldsymbol{k}' \cdot \boldsymbol{R}_2} 
        e^{i\boldsymbol{k} \cdot \boldsymbol{R}_3} 
        e^{i(\boldsymbol{k}'+\boldsymbol{q}') \cdot \boldsymbol{R}_4} 
        \left\langle i\boldsymbol{R}_1;j\boldsymbol{R}_2 \left| W \right| k\boldsymbol{R}_3;l\boldsymbol{R}_4 \right\rangle \\
      &= - \sum_{ijkl} U_{ic}^{\boldsymbol{k}+\boldsymbol{q}*} U_{jv'}^{\boldsymbol{k}'*} 
        U_{kv}^{\boldsymbol{k}} U_{lc'}^{\boldsymbol{k}'+\boldsymbol{q}'} 
        \sum_{\{\boldsymbol{R}\}} 
        e^{-i(\boldsymbol{k}+\boldsymbol{q}) \cdot \boldsymbol{R}_1} 
        e^{-i\boldsymbol{k}' \cdot \boldsymbol{R}_2} 
        e^{i\boldsymbol{k} \cdot \boldsymbol{R}_3} 
        e^{i(\boldsymbol{k}'+\boldsymbol{q}') \cdot \boldsymbol{R}_4} 
        W_{i\boldsymbol{R}_1,j\boldsymbol{R}_2,k\boldsymbol{R}_3,l\boldsymbol{R}_4} \\
      &= - \sum_{ijkl} U_{ic}^{\boldsymbol{k}+\boldsymbol{q}*} U_{jv'}^{\boldsymbol{k}'*} 
        U_{kv}^{\boldsymbol{k}} U_{lc'}^{\boldsymbol{k}'+\boldsymbol{q}'} 
        \sum_{\boldsymbol{R}} e^{i(\boldsymbol{q}'-\boldsymbol{q}) \cdot \boldsymbol{R}} 
        \sum_{\boldsymbol{R}_1\boldsymbol{R}_2\boldsymbol{R}_3} 
        e^{-i\boldsymbol{k}' \cdot \boldsymbol{R}_1} 
        e^{i\boldsymbol{k} \cdot \boldsymbol{R}_2} 
        e^{i(\boldsymbol{k}'+\boldsymbol{q}') \cdot \boldsymbol{R}_3} 
        W_{i\boldsymbol{0},j\boldsymbol{R}_1,k\boldsymbol{R}_2,l\boldsymbol{R}_3} \\
      &= - \frac{(2\pi)^3}{V} \delta(\boldsymbol{q}'-\boldsymbol{q}) 
        K_{vcv'c'}^{d \, \boldsymbol{q}}(\boldsymbol{k}, \boldsymbol{k}') \\
  \end{aligned}
\end{equation*}

\begin{equation}\label{Kd-wannier}
  K_{vcv'c'}^{d \, \boldsymbol{q}}(\boldsymbol{k}, \boldsymbol{k}') 
  = \sum_{ijkl} U_{ic}^{\boldsymbol{k}+\boldsymbol{q}*} U_{jv'}^{\boldsymbol{k}'*} 
      U_{kv}^{\boldsymbol{k}} U_{lc'}^{\boldsymbol{k}'+\boldsymbol{q}} 
    \sum_{\boldsymbol{R}_1\boldsymbol{R}_2\boldsymbol{R}_3} 
      e^{-i\boldsymbol{k}' \cdot \boldsymbol{R}_1} 
      e^{i\boldsymbol{k} \cdot \boldsymbol{R}_2} 
      e^{i(\boldsymbol{k}'+\boldsymbol{q}) \cdot \boldsymbol{R}_3} 
      W_{i\boldsymbol{0},j\boldsymbol{R}_1,k\boldsymbol{R}_2,l\boldsymbol{R}_3}
\end{equation}

\begin{equation*}
  \begin{aligned}
    K_{vcv'c'}^{x}(\boldsymbol{q}, \boldsymbol{k}, \boldsymbol{q}', \boldsymbol{k}') 
      &= \sum_{ijkl} U_{ic}^{\boldsymbol{k}+\boldsymbol{q}*} U_{jv'}^{\boldsymbol{k}'*} 
        U_{kc'}^{\boldsymbol{k}'+\boldsymbol{q}'} U_{lv}^{\boldsymbol{k}} 
        \left\langle i\boldsymbol{k}+\boldsymbol{q};j\boldsymbol{k}' \left| V \right| k\boldsymbol{k}'+\boldsymbol{q}';l\boldsymbol{k} \right\rangle \\
      &= \sum_{ijkl} U_{ic}^{\boldsymbol{k}+\boldsymbol{q}*} U_{jv'}^{\boldsymbol{k}'*} 
        U_{kc'}^{\boldsymbol{k}'+\boldsymbol{q}'} U_{lv}^{\boldsymbol{k}} 
        \sum_{\{\boldsymbol{R}\}} 
        e^{-i(\boldsymbol{k}+\boldsymbol{q}) \cdot \boldsymbol{R}_1} 
        e^{-i\boldsymbol{k}' \cdot \boldsymbol{R}_2} 
        e^{i(\boldsymbol{k}'+\boldsymbol{q}') \cdot \boldsymbol{R}_3} 
        e^{i\boldsymbol{k} \cdot \boldsymbol{R}_4} 
        \left\langle i\boldsymbol{R}_1;j\boldsymbol{R}_2 \left| V \right| k\boldsymbol{R}_3;l\boldsymbol{R}_4 \right\rangle \\
      &= \sum_{ijkl} U_{ic}^{\boldsymbol{k}+\boldsymbol{q}*} U_{jv'}^{\boldsymbol{k}'*} 
        U_{kc'}^{\boldsymbol{k}'+\boldsymbol{q}'} U_{lv}^{\boldsymbol{k}} 
        \sum_{\{\boldsymbol{R}\}} 
        e^{-i(\boldsymbol{k}+\boldsymbol{q}) \cdot \boldsymbol{R}_1} 
        e^{-i\boldsymbol{k}' \cdot \boldsymbol{R}_2} 
        e^{i(\boldsymbol{k}'+\boldsymbol{q}') \cdot \boldsymbol{R}_3} 
        e^{i\boldsymbol{k} \cdot \boldsymbol{R}_4} 
        V_{i\boldsymbol{R}_1,j\boldsymbol{R}_2,k\boldsymbol{R}_3,l\boldsymbol{R}_4} \\
      &= \sum_{ijkl} U_{ic}^{\boldsymbol{k}+\boldsymbol{q}*} U_{jv'}^{\boldsymbol{k}'*} 
        U_{kc'}^{\boldsymbol{k}'+\boldsymbol{q}'} U_{lv}^{\boldsymbol{k}} 
        \sum_{\boldsymbol{R}} e^{i(\boldsymbol{q}'-\boldsymbol{q}) \cdot \boldsymbol{R}} 
        \sum_{\boldsymbol{R}_1\boldsymbol{R}_2\boldsymbol{R}_3} 
        e^{-i\boldsymbol{k}' \cdot \boldsymbol{R}_1} 
        e^{i(\boldsymbol{k}'+\boldsymbol{q}') \cdot \boldsymbol{R}_2}
        e^{i\boldsymbol{k} \cdot \boldsymbol{R}_3} 
        V_{i\boldsymbol{0},j\boldsymbol{R}_1,k\boldsymbol{R}_2,l\boldsymbol{R}_3} \\
      &= \frac{(2\pi)^3}{V} \delta(\boldsymbol{q}'-\boldsymbol{q}) 
        K_{vcv'c'}^{x \, \boldsymbol{q}}(\boldsymbol{k}, \boldsymbol{k}') \\
  \end{aligned}
\end{equation*}

\begin{equation}\label{Kx-wannier}
  K_{vcv'c'}^{x \, \boldsymbol{q}}(\boldsymbol{k}, \boldsymbol{k}') 
  = \sum_{ijkl} U_{ic}^{\boldsymbol{k}+\boldsymbol{q}*} U_{jv'}^{\boldsymbol{k}'*} 
      U_{kc'}^{\boldsymbol{k}'+\boldsymbol{q}} U_{lv}^{\boldsymbol{k}} 
    \sum_{\boldsymbol{R}_1\boldsymbol{R}_2\boldsymbol{R}_3} 
      e^{-i\boldsymbol{k}' \cdot \boldsymbol{R}_1} 
      e^{i(\boldsymbol{k}'+\boldsymbol{q}) \cdot \boldsymbol{R}_2}
      e^{i\boldsymbol{k} \cdot \boldsymbol{R}_3} 
      V_{i\boldsymbol{0},j\boldsymbol{R}_1,k\boldsymbol{R}_2,l\boldsymbol{R}_3}
\end{equation}

\noindent
The Bethe-Salpeter equation becomes

\begin{equation}
  (E_{c\boldsymbol{k} + \boldsymbol{q}} - E_{v\boldsymbol{k}}) \phi_{vc}^{\alpha\boldsymbol{q}}(\boldsymbol{k}) 
  + \frac{V}{(2\pi)^3} \int \mathrm{d}\boldsymbol{k}' \sum_{v'c'} 
    K_{vc,v'c'}^{\boldsymbol{q}}(\boldsymbol{k},\boldsymbol{k}') 
    \phi_{v'c'}^{\alpha\boldsymbol{q}} (\boldsymbol{k}') 
  = E_{\alpha \boldsymbol{q}} \phi_{vc}^{\alpha\boldsymbol{q}}(\boldsymbol{k}) 
\end{equation}

For practical calculation, 
the continuous integration with respect to $\boldsymbol{k}$ has to be replaced by a summation of discrete grid in Brillouion Zone.
Actually this is the sampling scheme to calculate an integral in mathematics.
Let we define

\begin{equation}
  \begin{gathered}
    \frac{V}{(2\pi)^3} \int \mathrm{d}\boldsymbol{k} \sum_{vc} 
      \phi_{vc}^{\alpha \boldsymbol{q}*}(\boldsymbol{k}) \phi_{vc}^{\beta \boldsymbol{q}}(\boldsymbol{k}) 
      \approx \sum_{vc\boldsymbol{k}} \frac{V_{\boldsymbol{k}}}{V_{BZ}} 
        \phi_{vc}^{\alpha \boldsymbol{q}*}(\boldsymbol{k}) \phi_{vc}^{\beta \boldsymbol{q}}(\boldsymbol{k}) 
      = \sum_{vc\boldsymbol{k}} A_{vc\boldsymbol{k}}^{\alpha \boldsymbol{q}*} A_{vc\boldsymbol{k}}^{\beta \boldsymbol{q}}
      = \delta_{\alpha\beta} \\
    A_{vc\boldsymbol{k}}^{\alpha \boldsymbol{q}}
      = \sqrt{\frac{V_{\boldsymbol{k}}}{V_{BZ}}} \phi_{vc}^{\alpha \boldsymbol{q}}(\boldsymbol{k}) 
  \end{gathered}
\end{equation}

\begin{equation}
  \begin{gathered}
    (E_{c\boldsymbol{k} + \boldsymbol{q}} - E_{v\boldsymbol{k}}) \sqrt{\frac{V_{\boldsymbol{k}}}{V_{BZ}}} \phi_{vc}^{\alpha\boldsymbol{q}}(\boldsymbol{k}) 
      + \sum_{v'c'\boldsymbol{k}'} \sqrt{\frac{V_{\boldsymbol{k}}}{V_{BZ}}} 
      K_{vc,v'c'}^{\boldsymbol{q}}(\boldsymbol{k},\boldsymbol{k}') \sqrt{\frac{V_{\boldsymbol{k}'}}{V_{BZ}}} 
      \sqrt{\frac{V_{\boldsymbol{k}'}}{V_{BZ}}} \phi_{v'c'}^{\alpha\boldsymbol{q}} (\boldsymbol{k}') 
    = E_{\alpha \boldsymbol{q}} \sqrt{\frac{V_{\boldsymbol{k}}}{V_{BZ}}} \phi_{vc}^{\alpha\boldsymbol{q}}(\boldsymbol{k}) \\
    \Rightarrow (E_{c\boldsymbol{k} + \boldsymbol{q}} - E_{v\boldsymbol{k}}) A_{vc\boldsymbol{k}}^{\alpha\boldsymbol{q}} 
      + \sum_{v'c'\boldsymbol{k}'} K_{vc\boldsymbol{k},v'c'\boldsymbol{k}'}^{\boldsymbol{q}}
      A_{v'c'\boldsymbol{k}'}^{\alpha\boldsymbol{q}} 
    = E_{\alpha \boldsymbol{q}} A_{vc\boldsymbol{k}}^{\alpha\boldsymbol{q}} \\
    K_{vc\boldsymbol{k},v'c'\boldsymbol{k}'}^{\boldsymbol{q}} 
    = \sqrt{\frac{V_{\boldsymbol{k}}}{V_{BZ}}} K_{vc,v'c'}^{\boldsymbol{q}}(\boldsymbol{k},\boldsymbol{k}') \sqrt{\frac{V_{\boldsymbol{k}'}}{V_{BZ}}} 
  \end{gathered}
\end{equation}

\noindent
The weights $\sqrt{\frac{V_{\boldsymbol{k}}}{V_{BZ}}}$ is useful when using a group of irreducible points in Brillouion Zone.
Usually, we select an MonkhorstPack grid as the sampling scheme in the BZ. 
When we perform the calculation using the entire k-grid (that is, the reducible k-points), 
we have $\sqrt{\frac{V_{\boldsymbol{k}}}{V_{BZ}}} = \frac{1}{\sqrt{N}}$, where $N$ represents the number of sampling points.
In this situation we have $K_{vc\boldsymbol{k},v'c'\boldsymbol{k}'}^{\boldsymbol{q}} = \frac{1}{N} K_{vc,v'c'}^{\boldsymbol{q}}(\boldsymbol{k},\boldsymbol{k}')$.

\subsubsection{Kernal under UJ approximation}

The electron-hole interaction kernal under electronic basis have been demonstrated at \cref{Kd-wannier} and \cref{Kx-wannier}.
If we already get the hopping terms, we only need the interaction terms between wannier functions to calculate exciton state.
The interaction terms may be complicated, 
but we have confirmed that only keeping direct-terms and exchange-terms is a good approximation for most system with SU2 symmetry.

If we only keep the direct term and the exchange term, 

\begin{equation}
  \begin{aligned}
    W_{i\boldsymbol{0},j\boldsymbol{R}_1,k\boldsymbol{R}_2,l\boldsymbol{R}_3}
    &= \delta_{il}\delta_{jk}\delta_{\boldsymbol{R}_3\boldsymbol{0}}\delta_{\boldsymbol{R}_1\boldsymbol{R}_2} 
      W_{i\boldsymbol{0},j\boldsymbol{R}}
      +\delta_{ik}\delta_{jl}\delta_{\boldsymbol{R}_2\boldsymbol{0}}\delta_{\boldsymbol{R}_1\boldsymbol{R}_3} 
      J^{1}_{i\boldsymbol{0},j\boldsymbol{R}}
      +\delta_{ij}\delta_{kl}\delta_{\boldsymbol{R}_1\boldsymbol{0}}\delta_{\boldsymbol{R}_2\boldsymbol{R}_3} 
      J^{2}_{i\boldsymbol{0},k\boldsymbol{R}} \\
    V_{i\boldsymbol{0},j\boldsymbol{R}_1,k\boldsymbol{R}_2,l\boldsymbol{R}_3}
    &= \delta_{il}\delta_{jk}\delta_{\boldsymbol{R}_3\boldsymbol{0}}\delta_{\boldsymbol{R}_1\boldsymbol{R}_2} 
      V_{i\boldsymbol{0},j\boldsymbol{R}}
      +\delta_{ik}\delta_{jl}\delta_{\boldsymbol{R}_2\boldsymbol{0}}\delta_{\boldsymbol{R}_1\boldsymbol{R}_3} 
      X^{1}_{i\boldsymbol{0},j\boldsymbol{R}}
      +\delta_{ij}\delta_{kl}\delta_{\boldsymbol{R}_1\boldsymbol{0}}\delta_{\boldsymbol{R}_2\boldsymbol{R}_3} 
      X^{2}_{i\boldsymbol{0},k\boldsymbol{R}} \\
  \end{aligned}
\end{equation}

\noindent
With the definition $F_{ij}(\boldsymbol{k}) = \sum_{\boldsymbol{R}} e^{i\boldsymbol{k}\cdot\boldsymbol{R}} F_{i\boldsymbol{0},j\boldsymbol{R}}$,
we can simplify the Kernal to

\begin{equation}
  \begin{aligned}
    K_{vc\boldsymbol{k}v'c'\boldsymbol{k}'}^{d \, \boldsymbol{q}} = - \frac{1}{N}\sum_{ij}
      &\left[ U_{ic}^{\boldsymbol{k}+\boldsymbol{q}*} U_{jv'}^{\boldsymbol{k}'*} 
              U_{jv}^{\boldsymbol{k}} U_{ic'}^{\boldsymbol{k}'+\boldsymbol{q}} 
              W_{ij}(\boldsymbol{k}-\boldsymbol{k}') \right. \\
      &\left. + U_{ic}^{\boldsymbol{k}+\boldsymbol{q}*} U_{jv'}^{\boldsymbol{k}'*} 
                U_{iv}^{\boldsymbol{k}} U_{jc'}^{\boldsymbol{k}'+\boldsymbol{q}} 
                J_{ij}^{1}(\boldsymbol{q}) 
              + U_{ic}^{\boldsymbol{k}+\boldsymbol{q}*} U_{iv'}^{\boldsymbol{k}'*} 
                U_{jv}^{\boldsymbol{k}} U_{jc'}^{\boldsymbol{k}'+\boldsymbol{q}} 
                J_{ij}^{2}(\boldsymbol{k}+\boldsymbol{k}'+\boldsymbol{q}) 
      \right]
  \end{aligned}
\end{equation}

\begin{equation}
  \begin{aligned}
    K_{vc\boldsymbol{k}v'c'\boldsymbol{k}'}^{x \, \boldsymbol{q}} = \frac{1}{N}\sum_{ij}
      &\left[ U_{ic}^{\boldsymbol{k}+\boldsymbol{q}*} U_{jv'}^{\boldsymbol{k}'*} 
              U_{jc'}^{\boldsymbol{k}'+\boldsymbol{q}} U_{iv}^{\boldsymbol{k}} 
              V_{ij}(\boldsymbol{q}) \right. \\
      &\left. + U_{ic}^{\boldsymbol{k}+\boldsymbol{q}*} U_{jv'}^{\boldsymbol{k}'*} 
                U_{ic'}^{\boldsymbol{k}'+\boldsymbol{q}} U_{jv}^{\boldsymbol{k}} 
                X_{ij}^{1}(\boldsymbol{k}-\boldsymbol{k}') 
              + U_{ic}^{\boldsymbol{k}+\boldsymbol{q}*} U_{iv'}^{\boldsymbol{k}'*} 
                U_{jc'}^{\boldsymbol{k}'+\boldsymbol{q}} U_{jv}^{\boldsymbol{k}} 
                X_{ij}^{2}(\boldsymbol{k}+\boldsymbol{k}'+\boldsymbol{q}) 
      \right]
  \end{aligned}
\end{equation}

Here we emphasize that, direct term $W_{ij}(\boldsymbol{k})$ or $V_{ij}(\boldsymbol{k})$ contains the interaction over infinite distances
due to the $\frac{1}{r}$ decay trend of direct term.
Only in the infinite case we can see the summation of $\boldsymbol{R}$ contains the entire space clearly.
This is why we start from the BSE in a system with infinite size.



\subsubsection{Handle the divergency in Kernal}


\subsection{Excitonic Band state}

\subsubsection{Bloch state $\left| \alpha\boldsymbol{q} \right\rangle$}

Actually, we have give the Bloch states of exciton in \cref{Excitonic-Bloch-state} for infinite case:

\begin{equation}
  \begin{gathered}
    \left| \alpha \boldsymbol{q} \right\rangle 
    = \frac{V}{(2\pi)^3} \int \mathrm{d}\boldsymbol{k} \sum_{vc} \phi_{vc}^{\alpha \boldsymbol{q}}(\boldsymbol{k}) 
      \left| v c \boldsymbol{k}, \boldsymbol{q} \right\rangle \\
    \begin{aligned}
      \left\langle v c \boldsymbol{k}, \boldsymbol{q} \middle| v c \boldsymbol{k}', \boldsymbol{q}' \right\rangle
        &= \frac{(2\pi)^6}{V^2} \delta_{vv'} \delta_{cc'} \delta(\boldsymbol{k}-\boldsymbol{k}') \delta(\boldsymbol{q}-\boldsymbol{q}') \\
      \left\langle \alpha \boldsymbol{q} \middle| \beta \boldsymbol{q}' \right\rangle
        &= \frac{(2\pi)^3}{V} \delta_{\alpha\beta} \delta(\boldsymbol{q}-\boldsymbol{q}') 
    \end{aligned} \\
    \frac{V}{(2\pi)^3} \int \mathrm{d}\boldsymbol{k} \sum_{vc} 
      \phi_{vc}^{\alpha \boldsymbol{q}*}(\boldsymbol{k}) \phi_{vc}^{\beta \boldsymbol{q}}(\boldsymbol{k}) 
      = \delta_{\alpha\beta}
  \end{gathered}
\end{equation}

\noindent
Similarly we can give the definition for the approximation of finite case:

\begin{equation}
  \begin{gathered}
    \left| \alpha \boldsymbol{q} \right\rangle 
    = \sum_{vc\boldsymbol{k}} A_{vc\boldsymbol{k}}^{\alpha \boldsymbol{q}}
      \left| v c \boldsymbol{k}, \boldsymbol{q} \right\rangle \\
    \begin{aligned}
      \left\langle v c \boldsymbol{k}, \boldsymbol{q} \middle| v c \boldsymbol{k}', \boldsymbol{q}' \right\rangle
        &= \frac{(2\pi)^3}{V} \delta_{vv'} \delta_{cc'} \delta_{\boldsymbol{k},\boldsymbol{k}'} \delta(\boldsymbol{q}-\boldsymbol{q}') \\
      \left\langle \alpha \boldsymbol{q} \middle| \beta \boldsymbol{q}' \right\rangle
        &= \frac{(2\pi)^3}{V} \delta_{\alpha\beta} \delta(\boldsymbol{q}-\boldsymbol{q}') 
    \end{aligned} \\
    \sum_{vc\boldsymbol{k}} A_{vc\boldsymbol{k}}^{\alpha \boldsymbol{q}*} A_{vc\boldsymbol{k}}^{\beta \boldsymbol{q}}
      = \delta_{\alpha\beta}
  \end{gathered}
\end{equation}

\noindent
Note we retain $\frac{(2\pi)^3}{V}\delta(\boldsymbol{q}-\boldsymbol{q}')$ because we regard the finite case as the approximation of infinite case.
This notion allows us to calculate the excitonic states of a random $\boldsymbol{q}$ by solving BSE on a finite grid in BZ numerically.



\subsubsection{$\left| u_{\alpha\boldsymbol{q}} \right\rangle = e^{-i\boldsymbol{q} \cdot \hat{\boldsymbol{r}}_0} \left| \alpha\boldsymbol{q} \right\rangle$}

Exciton have two spacial varialbes: $\boldsymbol{r}_e$ and $\boldsymbol{r}_h$.
We can define $\lambda$ as the weight to calculate the center position of exciton:

\begin{equation}
  \begin{aligned}
    \boldsymbol{r}_0 = (1-\lambda) \boldsymbol{r}_e + \lambda \boldsymbol{r}_h ,\, 
    \boldsymbol{r} = \boldsymbol{r}_e-\boldsymbol{r}_h \\
    \boldsymbol{r}_e = \boldsymbol{r}_0 + \lambda \boldsymbol{r} ,\,
    \boldsymbol{r}_h = \boldsymbol{r}_0 - (1-\lambda) \boldsymbol{r} 
  \end{aligned}
\end{equation}

\noindent
Define:

\begin{equation}
  \begin{gathered}
    \left| u_{\alpha\boldsymbol{q}}^{\lambda} \right\rangle 
    = e^{-i\boldsymbol{q} \cdot \hat{\boldsymbol{r}}_0} \left| \alpha\boldsymbol{q} \right\rangle 
    = \frac{V}{(2\pi)^3} \int \mathrm{d}\boldsymbol{k} \sum_{vc} \phi_{vc}^{\alpha \boldsymbol{q}}(\boldsymbol{k}) 
      e^{-i\boldsymbol{q} \cdot \hat{\boldsymbol{r}}_0} \left| v c \boldsymbol{k}, \boldsymbol{q} \right\rangle
    = \frac{V}{(2\pi)^3} \int \mathrm{d}\boldsymbol{k} \sum_{vc} \phi_{vc}^{\alpha \boldsymbol{q}}(\boldsymbol{k}) 
      \left| u_{v c \boldsymbol{k}}^{\lambda\boldsymbol{q}} \right\rangle \\
    \left| u_{v c \boldsymbol{k}}^{\lambda\boldsymbol{q}} \right\rangle 
    = e^{-i\boldsymbol{q} \cdot \hat{\boldsymbol{r}}_0} \left| v c \boldsymbol{k}, \boldsymbol{q} \right\rangle 
    = e^{-i\boldsymbol{q} \cdot \hat{\boldsymbol{r}}_0} 
      \left| \psi_{c\boldsymbol{k}+\boldsymbol{q}}^{e} \right\rangle \left| \psi_{v\boldsymbol{k}}^{h} \right\rangle 
    = e^{i(\boldsymbol{k}+\lambda\boldsymbol{q}) \cdot \hat{\boldsymbol{r}}} 
      \left| u_{c\boldsymbol{k}+\boldsymbol{q}}^{e} \right\rangle \left| u_{v\boldsymbol{k}}^{h} \right\rangle 
  \end{gathered}
\end{equation}

\noindent
Under translation operator $\hat{T}_{\boldsymbol{R}}$, we have

\begin{equation}
  \begin{gathered}
    \hat{T}_{\boldsymbol{R}} \left| v c \boldsymbol{k}, \boldsymbol{q} \right\rangle 
    = e^{i(\boldsymbol{k} + \boldsymbol{q})\cdot\boldsymbol{R}}e^{-i\boldsymbol{k}\cdot\boldsymbol{R}} 
      \left| v c \boldsymbol{k}, \boldsymbol{q} \right\rangle 
    = e^{i\boldsymbol{q}\cdot\boldsymbol{R}} 
      \left| v c \boldsymbol{k}, \boldsymbol{q} \right\rangle 
    \Rightarrow \hat{T}_{\boldsymbol{R}} \left| \alpha\boldsymbol{q} \right\rangle 
    = e^{i\boldsymbol{q}\cdot\boldsymbol{R}} \left| \alpha\boldsymbol{q} \right\rangle \\
    \hat{T}_{\boldsymbol{R}} \left| u_{\alpha\boldsymbol{q}}^{\lambda} \right\rangle 
    = e^{-i\boldsymbol{q} \cdot (\hat{\boldsymbol{r}}_0 + \boldsymbol{R})} e^{i\boldsymbol{q}\cdot\boldsymbol{R}} \left| \alpha\boldsymbol{q} \right\rangle 
    = \left| u_{\alpha\boldsymbol{q}}^{\lambda} \right\rangle 
  \end{gathered}
\end{equation}

\noindent
This prove that $\left| u_{\alpha\boldsymbol{q}}^{\lambda} \right\rangle$ is a periodic function defined in an unitcell.


Similarly, we need to calculate the inner product of $\left\langle u_{\alpha\boldsymbol{q}}^{\lambda} \middle| u_{\beta\boldsymbol{q}'}^{\lambda} \right\rangle$ in an unitcell,
where $\boldsymbol{q}$ and $\boldsymbol{q}'$ is adjacent.
We try to analyze the inner product of basis $\left| u_{v c \boldsymbol{k}}^{\lambda\boldsymbol{q}} \right\rangle$ firstly.

\begin{equation}
  \begin{aligned}
    \lim_{\boldsymbol{q} \rightarrow \boldsymbol{q}'} 
    \left\langle u_{v c \boldsymbol{k}}^{\lambda\boldsymbol{q}} \middle| u_{v' c' \boldsymbol{k}'}^{\lambda\boldsymbol{q}'} \right\rangle_{uc} 
    &= \frac{1}{\sum_{\boldsymbol{R}}} 
      \left\langle u_{c\boldsymbol{k}+\boldsymbol{q}}^{e} \right| \left\langle u_{v\boldsymbol{k}}^{h} \right| 
      e^{i(\boldsymbol{k}'+\lambda\boldsymbol{q}' -\boldsymbol{k}-\lambda\boldsymbol{q}) \cdot \hat{\boldsymbol{r}}} 
      \left| u_{c'\boldsymbol{k}'+\boldsymbol{q}'}^{e} \right\rangle \left| u_{v'\boldsymbol{k}'}^{h} \right\rangle \\
    &= \frac{1}{\sum_{\boldsymbol{R}}} 
      \left\langle u_{c\boldsymbol{k}+\boldsymbol{q}}^{e} \right|
      e^{i(\boldsymbol{k}'+\lambda\boldsymbol{q}' -\boldsymbol{k}-\lambda\boldsymbol{q}) \cdot \hat{\boldsymbol{r}}_e} 
      \left| u_{c'\boldsymbol{k}'+\boldsymbol{q}'}^{e} \right\rangle 
      \left\langle u_{v\boldsymbol{k}}^{h} \right| 
      e^{i(\boldsymbol{k}'+\lambda\boldsymbol{q}' -\boldsymbol{k}-\lambda\boldsymbol{q}) \cdot \hat{\boldsymbol{r}}_h} 
      \left| u_{v'\boldsymbol{k}'}^{h} \right\rangle \\
    &= \frac{1}{\sum_{\boldsymbol{R}}} 
      \sum_{\boldsymbol{R}} 
      e^{i(\boldsymbol{k}'+\lambda\boldsymbol{q}' -\boldsymbol{k}-\lambda\boldsymbol{q}) \cdot \boldsymbol{R}} 
      \left\langle u_{c\boldsymbol{k}+\boldsymbol{q}}^{e} \right|
      e^{i(\boldsymbol{k}'+\lambda\boldsymbol{q}' -\boldsymbol{k}-\lambda\boldsymbol{q}) \cdot \hat{\boldsymbol{r}}_e} 
      \left| u_{c'\boldsymbol{k}'+\boldsymbol{q}'}^{e} \right\rangle_{uc} 
      \left\langle u_{v\boldsymbol{k}}^{h} \right| 
      e^{i(\boldsymbol{k}'+\lambda\boldsymbol{q}' -\boldsymbol{k}-\lambda\boldsymbol{q}) \cdot \hat{\boldsymbol{r}}_h} 
      \left| u_{v'\boldsymbol{k}'}^{h} \right\rangle \\
    &= \frac{1}{\sum_{\boldsymbol{R}}} 
      \frac{(2\pi)^3}{V} \delta(\boldsymbol{k}'+\lambda\boldsymbol{q}' -\boldsymbol{k}-\lambda\boldsymbol{q})
      \left\langle u_{c\boldsymbol{k}+\boldsymbol{q}}^{e} \middle| u_{c'\boldsymbol{k}'+\boldsymbol{q}'}^{e} \right\rangle_{uc} 
      \left\langle u_{v\boldsymbol{k}}^{h} \middle| u_{v'\boldsymbol{k}'}^{h} \right\rangle \\
    &= \frac{(2\pi)^3}{V} \delta(\boldsymbol{k}'+\lambda\boldsymbol{q}' -\boldsymbol{k}-\lambda\boldsymbol{q})
      \left\langle u_{c\boldsymbol{k}+\boldsymbol{q}}^{e} \middle| u_{c'\boldsymbol{k}'+\boldsymbol{q}'}^{e} \right\rangle_{uc} 
      \left\langle u_{v\boldsymbol{k}}^{h} \middle| u_{v'\boldsymbol{k}'}^{h} \right\rangle_{uc} \\
  \end{aligned}
\end{equation}

\noindent
Here, the $\delta$-function means we have to perform the conditional summation of $A_{vc\boldsymbol{k}}^{\alpha\boldsymbol{q}}$ 
based on the specific values of $\boldsymbol{k}$ and $\boldsymbol{q}$ in numerical calculation.
It's a little bit complicated.
But we find another method to give a concrete form representing $\left| u_{\alpha\boldsymbol{q}} \right\rangle$:

\begin{equation}
  \begin{aligned}
    \lim_{\boldsymbol{q} \rightarrow \boldsymbol{q}'} 
    \left\langle u_{v c \boldsymbol{k}}^{\lambda\boldsymbol{q}} \middle| 
    u_{v' c' \boldsymbol{k}'}^{\lambda\boldsymbol{q}'} \right\rangle_{uc} 
    &= \frac{1}{\sum_{\boldsymbol{R}}} 
      \left\langle \psi_{c\boldsymbol{k}+\boldsymbol{q}}^{e} \right| \left\langle \psi_{v\boldsymbol{k}}^{h} \right| 
        e^{i(\boldsymbol{q} - \boldsymbol{q}') \cdot [(1-\lambda)\hat{\boldsymbol{r}}_e + \lambda\hat{\boldsymbol{r}}_h]} 
      \left| \psi_{c'\boldsymbol{k}'+\boldsymbol{q}'}^{e} \right\rangle \left| \psi_{v'\boldsymbol{k}'}^{h} \right\rangle \\
    &= \frac{1}{\sum_{\boldsymbol{R}}} 
      \left\langle \psi_{c\boldsymbol{k}+\boldsymbol{q}}^{e} \right| 
        e^{i(1-\lambda)(\boldsymbol{q} - \boldsymbol{q}') \cdot \hat{\boldsymbol{r}}_e} 
      \left| \psi_{c'\boldsymbol{k}'+\boldsymbol{q}'}^{e} \right\rangle 
      \left\langle \psi_{v\boldsymbol{k}}^{h} \right| 
        e^{i\lambda(\boldsymbol{q} - \boldsymbol{q}') \cdot \hat{\boldsymbol{r}}_h} 
      \left| \psi_{v'\boldsymbol{k}'}^{h} \right\rangle \\
    &= \frac{1}{\sum_{\boldsymbol{R}}} \sum_{ij\boldsymbol{R}_e\boldsymbol{R}_h}\sum_{i'j'\boldsymbol{R}_e'\boldsymbol{R}_h'}
      U_{ic}^{\boldsymbol{k}+\boldsymbol{q}*} U_{jv}^{\boldsymbol{k}} 
      U_{i'c'}^{\boldsymbol{k}'+\boldsymbol{q}'} U_{j'v'}^{\boldsymbol{k}'*} 
      e^{-i(\boldsymbol{k}+\boldsymbol{q}) \cdot \boldsymbol{R}_e} 
      e^{i\boldsymbol{k} \cdot \boldsymbol{R}_h}
      e^{i(\boldsymbol{k}'+\boldsymbol{q}') \cdot \boldsymbol{R}_e'}
      e^{-i\boldsymbol{k}' \cdot \boldsymbol{R}_h'} \\
    &\phantom{=\frac{1}{\sum_{R}} \sum_{ij\boldsymbol{R}_e\boldsymbol{R}_h}\sum_{i'j'\boldsymbol{R}_e'\boldsymbol{R}_h'}} 
      \times \left\langle \psi_{i\boldsymbol{R}_e}^{e} \right| 
        e^{i(1-\lambda)(\boldsymbol{q} - \boldsymbol{q}') \cdot \hat{\boldsymbol{r}}_e} 
      \left| \psi_{i'\boldsymbol{R}_e'}^{e} \right\rangle 
      \left\langle \psi_{j\boldsymbol{R}_h}^{h} \right| 
        e^{i\lambda(\boldsymbol{q} - \boldsymbol{q}') \cdot \hat{\boldsymbol{r}}_h} 
      \left| \psi_{j'\boldsymbol{R}_h'}^{h} \right\rangle 
  \end{aligned}
\end{equation}

\noindent
If electronic wannier function is MLWF, we have

\begin{equation}
  \lim_{\boldsymbol{k} \rightarrow 0} \left\langle \psi_{i\boldsymbol{R}}^{} \right| 
      e^{i\boldsymbol{k} \cdot \hat{\boldsymbol{r}}} 
    \left| \psi_{i'\boldsymbol{R}'}^{} \right\rangle 
  = \delta_{ii'} \delta_{\boldsymbol{R}\boldsymbol{R}'} e^{i\boldsymbol{k} \cdot (\boldsymbol{R} + \boldsymbol{\tau}_i)} 
\end{equation}

\noindent
then

\begin{equation}
  \begin{aligned}
    \lim_{\boldsymbol{q} \rightarrow \boldsymbol{q}'} 
    \left\langle u_{v c \boldsymbol{k}}^{\lambda\boldsymbol{q}} \middle| 
    u_{v' c' \boldsymbol{k}'}^{\lambda\boldsymbol{q}'} \right\rangle_{uc} 
    &= \frac{1}{\sum_{\boldsymbol{R}}} \sum_{ij\boldsymbol{R}_e\boldsymbol{R}_h}
      U_{ic}^{\boldsymbol{k}+\boldsymbol{q}*} U_{jv}^{\boldsymbol{k}} 
      U_{ic'}^{\boldsymbol{k}'+\boldsymbol{q}'} U_{jv'}^{\boldsymbol{k}'*} 
      e^{-i(\boldsymbol{k}+\boldsymbol{q}) \cdot \boldsymbol{R}_e} 
      e^{i\boldsymbol{k} \cdot \boldsymbol{R}_h}
      e^{i(\boldsymbol{k}'+\boldsymbol{q}') \cdot \boldsymbol{R}_e}
      e^{-i\boldsymbol{k}' \cdot \boldsymbol{R}_h}\\
    &\phantom{=\frac{1}{\sum_{R}} \sum_{ij\boldsymbol{R}_e\boldsymbol{R}_h}} 
      \times e^{i(1-\lambda)(\boldsymbol{q} - \boldsymbol{q}') \cdot (\boldsymbol{R}_e + \boldsymbol{\tau}_i)} 
        e^{i\lambda(\boldsymbol{q} - \boldsymbol{q}') \cdot (\boldsymbol{R}_h + \boldsymbol{\tau}_j)} \\
    &= \frac{1}{\sum_{\boldsymbol{R}}} \sum_{ij\boldsymbol{R}_e\boldsymbol{R}_h}
      U_{ic}^{\boldsymbol{k}+\boldsymbol{q}*} U_{jv}^{\boldsymbol{k}} 
      U_{ic'}^{\boldsymbol{k}'+\boldsymbol{q}'} U_{jv'}^{\boldsymbol{k}'*} 
      e^{i(\boldsymbol{k} - \boldsymbol{k}') \cdot (\boldsymbol{R}_h - \boldsymbol{R}_e)}
      e^{i\lambda(\boldsymbol{q} - \boldsymbol{q}') \cdot (\boldsymbol{R}_h - \boldsymbol{R}_e)} \\
    &\phantom{=\frac{1}{\sum_{R}} \sum_{ij\boldsymbol{R}_e\boldsymbol{R}_h}} 
      \times e^{i(\boldsymbol{q} - \boldsymbol{q}') \cdot [(1-\lambda)\boldsymbol{\tau}_i + \lambda\boldsymbol{\tau}_j]} \\
    &= \frac{1}{\sum_{\boldsymbol{R}}} \sum_{\boldsymbol{R}_0}\sum_{ij\boldsymbol{R}}
      U_{ic}^{\boldsymbol{k}+\boldsymbol{q}*} U_{jv}^{\boldsymbol{k}} 
      U_{ic'}^{\boldsymbol{k}'+\boldsymbol{q}'} U_{jv'}^{\boldsymbol{k}'*} 
      e^{i(\boldsymbol{k} - \boldsymbol{k}') \cdot \boldsymbol{R}}
      e^{i\lambda(\boldsymbol{q} - \boldsymbol{q}') \cdot \boldsymbol{R}}
      e^{i(\boldsymbol{q} - \boldsymbol{q}') \cdot [(1-\lambda)\boldsymbol{\tau}_i + \lambda\boldsymbol{\tau}_j]} \\
    &= \sum_{ij\boldsymbol{R}}
      U_{ic}^{\boldsymbol{k}+\boldsymbol{q}*} U_{jv}^{\boldsymbol{k}} 
      U_{ic'}^{\boldsymbol{k}'+\boldsymbol{q}'} U_{jv'}^{\boldsymbol{k}'*} 
      e^{i(\boldsymbol{k} - \boldsymbol{k}') \cdot \boldsymbol{R}}
      e^{i\lambda(\boldsymbol{q} - \boldsymbol{q}') \cdot \boldsymbol{R}}
      e^{i(\boldsymbol{q} - \boldsymbol{q}') \cdot [(1-\lambda)\boldsymbol{\tau}_i + \lambda\boldsymbol{\tau}_j]} \\
    &= \sum_{ij\boldsymbol{R}} A_{ij\boldsymbol{R}}^{vc\boldsymbol{k},\boldsymbol{q}*} A_{ij\boldsymbol{R}}^{v'c'\boldsymbol{k}',\boldsymbol{q}'}
  \end{aligned}
\end{equation}

\noindent
where

\begin{equation}
    \begin{gathered}
        A_{ij\boldsymbol{R}}^{vc\boldsymbol{k},\boldsymbol{q}}
        = U_{ic}^{\boldsymbol{k}+\boldsymbol{q}} U_{jv}^{\boldsymbol{k}*} 
            e^{-i (\boldsymbol{k} + \lambda \boldsymbol{q}) \cdot \boldsymbol{R}}
            e^{- i\boldsymbol{q} \cdot [(1-\lambda)\boldsymbol{\tau}_i + \lambda\boldsymbol{\tau}_j]} \\
        \sum_{ij\boldsymbol{R}} A_{ij\boldsymbol{R}}^{vc\boldsymbol{k},\boldsymbol{q}*} A_{ij\boldsymbol{R}}^{vc\boldsymbol{k},\boldsymbol{q}} = 1
    \end{gathered}
\end{equation}

\noindent
We can use $A_{ij\boldsymbol{R}}^{vc\boldsymbol{k}}$ to represent $\left| u_{vc\boldsymbol{k}}^{\lambda\boldsymbol{q}} \right\rangle$.
For excitonic band state we have

\begin{equation}
  \begin{aligned}
    A_{ij\boldsymbol{R}}^{\alpha\boldsymbol{q}}
    &= e^{-i\lambda \boldsymbol{q} \cdot \boldsymbol{R}}
      e^{- i\boldsymbol{q} \cdot [(1-\lambda)\boldsymbol{\tau}_i + \lambda\boldsymbol{\tau}_j]} 
      \frac{V}{(2\pi)^3} \int \mathrm{d}\boldsymbol{k} \sum_{vc} \phi_{vc}^{\alpha \boldsymbol{q}}(\boldsymbol{k}) 
      U_{ic}^{\boldsymbol{k}+\boldsymbol{q}} U_{jv}^{\boldsymbol{k}*} 
      e^{-i \boldsymbol{k} \cdot \boldsymbol{R}} \\
    &\approx e^{-i\lambda \boldsymbol{q} \cdot \boldsymbol{R}}
      e^{- i\boldsymbol{q} \cdot [(1-\lambda)\boldsymbol{\tau}_i + \lambda\boldsymbol{\tau}_j]} 
      \frac{1}{\sqrt{N}} \sum_{vc\boldsymbol{k}} A_{vc\boldsymbol{k}}^{\alpha \boldsymbol{q}} 
      U_{ic}^{\boldsymbol{k}+\boldsymbol{q}} U_{jv}^{\boldsymbol{k}*} 
      e^{-i \boldsymbol{k} \cdot \boldsymbol{R}} 
  \end{aligned}
\end{equation}

There is only one thing need to be noticed. 
$\boldsymbol{R}$ actually represents the distance between the cells where electron or hole locates. 
The bigger $|\boldsymbol{R}|$ is, the smaller $|A_{ij\boldsymbol{R}}^{\alpha\boldsymbol{q}}|$ is.
In infinite case, the value of $\boldsymbol{R}$ is infinitely numerous. 
But in finite case, $\boldsymbol{R}$ has only $N$ possible values.
Here $N$ is the number of kpoints and represents the size of system.
So we have to ensure that $N$ is large enough 
so that $A_{ij\boldsymbol{R}}^{\alpha\boldsymbol{q}}$ includes all the sufficiently large components.

\subsection{Excitonic Topology}

This subsection only give the simple deduce, then refer other paper.

\subsubsection{Berry Connection}

\begin{equation}
    \begin{aligned}
        \mathcal{A}_{\alpha \beta}^{0}(q) 
        &= \frac{V}{(2\pi)^3} \int \mathrm{d}k \sum_{vc} \left[
            i \phi_{vc}^{\alpha \boldsymbol{q}*}(\boldsymbol{k}) \boldsymbol{\nabla}_{\boldsymbol{q}} 
            \phi_{vc}^{\beta \boldsymbol{q}} (\boldsymbol{k}) 
            +\sum_{c'} A_{vck}^{\alpha q*} A_{vc'k}^{\beta q} A_{cc'}^{ele}(k+q) 
            \right] \\
        A_{\alpha \beta}^{1}(q) 
        &= \frac{V}{(2\pi)^3} \int \mathrm{d}k \sum_{vc} \left[
            i A_{vck}^{\alpha q*} \nabla_{q} A_{vck}^{\beta q} 
            -i A_{vck}^{\alpha q*} \nabla_{k} A_{vck}^{\beta q} 
            +\sum_{v'} A_{vck}^{\alpha q*} A_{v'ck}^{\beta q} A_{v'v}^{ele}(k) 
            \right] 
    \end{aligned}
\end{equation}

\begin{equation}
    A_{\alpha \beta}^{\lambda}(q) 
    = i \left\langle u_{\alpha q}^{\lambda} \right| \nabla_{q} \left| u_{\beta q}^{\lambda} \right\rangle_{uc} 
    = (1-\lambda) A_{\alpha \beta}^{0}(q) + \lambda A_{\alpha \beta}^{1}(q) 
\end{equation}


\subsubsection{Berry Curvature}


