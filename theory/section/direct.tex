\section{Direct term between wannier functions}

The direct term between wannier functions is defined as

\begin{equation}
    \begin{gathered}
        V_{ij}(\boldsymbol{k}) = \sum_{\boldsymbol{R}} e^{i\boldsymbol{k}\cdot\boldsymbol{R}} V_{i\boldsymbol{0},j\boldsymbol{R}} \\
        V_{i\boldsymbol{0},j\boldsymbol{R}} 
            = \iint \mathrm{d} \boldsymbol{r} \mathrm{d} \boldsymbol{r}'
            \left| \psi_{i\boldsymbol{0}}^{}(\boldsymbol{r}) \right|^{2} 
                V(\boldsymbol{r},\boldsymbol{r}') 
            \left| \psi_{j\boldsymbol{R}}^{}(\boldsymbol{r}') \right|^{2} 
    \end{gathered}
\end{equation}

\subsection{Long-range part}

With the Coulomb potential generated by a charge of Gaussian distribution \eqref{Gaussian-potential},
we can expand $V_{ij}(\boldsymbol{k})$ as

\begin{equation}
  \begin{aligned}
    V_{ij}(\boldsymbol{k}) 
    &= \sum_{\boldsymbol{R}} e^{i\boldsymbol{k}\cdot\boldsymbol{R}} V_{i\boldsymbol{0},j\boldsymbol{R}} \\
    &= \sum_{\boldsymbol{R}} e^{i\boldsymbol{k}\cdot\boldsymbol{R}} \left[
        V_{i\boldsymbol{0},j\boldsymbol{R}} - \phi(\boldsymbol{R} + \boldsymbol{\tau}_j - \boldsymbol{\tau}_i)
      \right] 
      + \sum_{\boldsymbol{R}} e^{i\boldsymbol{k}\cdot\boldsymbol{R}} 
        \phi(\boldsymbol{R} + \boldsymbol{\tau}_j - \boldsymbol{\tau}_i) \\
    &= V_{ij}^{sr}(\boldsymbol{k}) + \phi_{ij}(\boldsymbol{k})
  \end{aligned}
\end{equation}

\noindent
The components of short-range term $V_{ij}^{sr}(\boldsymbol{k})$ decay to $0$ rapidly, 
so that $V_{ij}^{sr}(\boldsymbol{k})$ can be calculated in real space.
With fourier transformation \eqref{}, the long-range term can be expressed



In this section, we will give a analytical form of the long range part of direct term. 
We use the screened interaction as example, the naked direct term is similar.
The direct terms between wannier functions is defined as

\begin{equation}
  W_{i\boldsymbol{0},j\boldsymbol{R}} 
  = \iint \mathrm{d} \boldsymbol{r} \mathrm{d} \boldsymbol{r}'
      \left| \psi_{i\boldsymbol{0}}^{}(\boldsymbol{r}) \right|^{2} 
      W(\boldsymbol{r},\boldsymbol{r}') 
      \left| \psi_{j\boldsymbol{R}}^{}(\boldsymbol{r}') \right|^{2} 
\end{equation}



\noindent
When $\boldsymbol{R}$ is large enough, 
$W_{i\boldsymbol{0},j\boldsymbol{R}}$ is approximate to $W(\boldsymbol{R} + \boldsymbol{\tau}_j - \boldsymbol{\tau}_i)$.
The components of short-range term $W_{ij}^{sr}(\boldsymbol{k})$ decay to $0$ rapidly, 
so that $W_{ij}^{sr}(\boldsymbol{k})$ can be calculated in real space.
With Poisson summation, the long-range term can be expressed

\begin{equation}
  W_{ij}^{lr}(\boldsymbol{k}) = \frac{1}{V} \sum_{\boldsymbol{G}} W(\boldsymbol{k} + \boldsymbol{G}) 
    e^{-i(\boldsymbol{k} + \boldsymbol{G}) \cdot (\boldsymbol{\tau}_j - \boldsymbol{\tau}_i)}
\end{equation}



The number of parameters that we can provide is a finite set, and we 


\subsection{Mirror Correction}

